\documentclass[stu,12pt,floatsintext,longtable]{apa7}

\usepackage[T1]{fontenc} 
\usepackage[utf8]{inputenc}
\usepackage[ngerman]{babel}

\usepackage[babel,german=quotes]{csquotes} 
\usepackage[style=apa,sortcites=true,sorting=nyt,backend=biber,language=german]{biblatex}
% \addbibresource{report/references/bibliography.bib}
\addbibresource{report/references/references-new.bib}
\DeclareLanguageMapping{german}{german-apa}
\usepackage{mathptmx}




%%% --- Von No eingebundene Pakete


% Grafiken und Tabellen
\usepackage{lscape}
\usepackage{graphicx}
\usepackage{tabularx}
\usepackage{xltabular}
\usepackage{booktabs}
\usepackage{ragged2e}
\usepackage{multicol}
\usepackage{multirow}
% \usepackage{longtable}
\usepackage{pdfpages}
% \UseTblrLibrary{booktabs}
% Laden verschiedener Pakete für Querverweise
\usepackage{enumitem}
\usepackage{varioref}
\usepackage{hyperref}
\usepackage[ngerman]{cleveref}

% Deutsche Aliase für Tabellen
\crefname{table}{Tabelle}{Tabellen}
\Crefname{table}{Tabelle}{Tabellen}

% Laden des XColor-Pakets
\usepackage[table]{xcolor}
\newcommand{\td}{\textcolor{red}{\textbf{???}}{ }}

\title{Untersuchung der theoretischen und methodischen Fundierung von \protect\textcite{brotzeller_exploring_2025} und ein Reproduktionsversuch} % The big, long version of the title for the title page
\shorttitle{Asymmetric Self Concept Change} % The short title for the header
\authorsnames{Benedikt Daniel Ehrenwirth\\Matrikelnummer: }
\note{Zeichenzahl: \td}
\duedate{13. Januar 2026}
% \date{January 17, 2024} The student version doesn't use the \date command, for whatever reason
\authorsaffiliations{Ludwig-Maximilians-Universität}
\course{Bachelorarbeit} % LaTeX gets annoyed (i.e., throws a grumble-error) if this is blank, so I put something here. However, if your instructor will mark you off for this being on the title page, you can leave this entry blank (delete the PSY 4321, but leave the command), and just make peace with the error that will happen. It won't break the document.
\professor{Dr. Philipp Sckopke}  % Same situation as for the course info. Some instructors want this, some absolutely don't and will take off points. So do what you gotta.

\abstract{
Die Replikationskrise in der Psychologie macht die Entwicklung besserer Theorien erforderlich. Nur wenn eine Theorie präzise und kohärent ist, lassen sich aus ihr exakte Vorhersagen ableiten. Ausschließlich verbal beschriebene Theorien können den Präzisions- und Kohärenzanspruch in der Regel nicht erfüllen und müssen um eine Formalisierung ergänzt werden. Neben der besseren Ableitbarkeit von Vorhersagen dient die Formalisierung dazu, beurteilen zu können, ob eine Theorie das vorhergesagte Phänomen auch plausibel erklären kann. Die vorliegende Arbeit versucht am Beispiel der Überprüfung der Theorie von \textcite{brotzeller_exploring_2025} aufzuzeigen, dass die Güte einer Theorie nur durch eine Kombination aus der Beurteilung ihrer Vorhersage- und ihrer Erklärungskraft bewertet werden kann.

Die theoretischen Annahmen werden formalisiert und dabei auf ihre Präzision und Kohärenz geprüft. Im Rahmen einer Re-Implementation-Reproduction \autocite{kohrt_conceptual_2024} wird versucht, die Ergebnisse zu reproduzieren, mit denen die Autoren auf eine geringe Vorhersagekraft der Theorie schließen.

Nach der Definition des Productive-Explanation-Framework \autocite{van_dongen_productive_2025} weist die Theorie eine geringe Erklärungskraft auf. Die Ergebnisse sind reproduzierbar, was die Schlussfolgerung auf eine geringe Vorhersagekraft der Theorie unterstützt. Allerdings kann nicht ausgeschlossen werden, dass die getroffene Vorhersage eine notwendige Folge der Theorie ist, weshalb die Schlussfolgerung zu keinem abschließenden Urteil kommen kann.

Die Formalisierung von Theorien fördert die Entwicklung besserer Theorien, indem sie die Beurteilung ihrer Erklärungskraft ermöglicht und der Überprüfung ihre Vorhersagekraft zu einer höheren Aussagekraft verhilft.
}

\begin{document}
    \maketitle

    \tableofcontents
    \clearpage
    \listoffigures
    \clearpage
    \listoftables

    % Kapitel
    \section{Einleitung}
Die Replikationskrise in der Psychologie gab Anlass zu vielen Gegenmaßnahmen, mit denen die psychologische Forschung verbessert werden sollte. Auch wenn diese Maßnahmen in erster Linie die Qualität der empirischen Methodik betrafen, wurden in jüngster Zeit Stimmen lauter, welche die geringe Qualität psychologischer Theorien mit der Replikationskrise in Verbindung bringen und die Verbesserung der Theoriearbeit fordern \autocites{borsboom_theory_2021,van_dongen_productive_2025}. Die psychologische Forschung konzentriert sich vor allem auf die empirische Überprüfung von Vorhersagen und vernachlässigt die systematische Entwicklung guter Theorien, obwohl gute Theorien die Voraussetzung dafür sind, präzise Vorhersagen zu treffen \autocite{borsboom_theory_2021}. Soll die Vorhersage eines Phänomens der Verifizierung einer Theorie dienen, dann ist das Eintreten der Vorhersage nur dann eine valide Aussage über die Gültigkeit der Theorie, wenn die Theorie tatsächlich diese Vorhersage trifft. Umgekehrt gilt: Macht eine Theorie eine bestimmte Vorhersage und tritt diese in der Realität nicht ein, falsifiziert die fehlende Beobachtung der Vorhersage die Theorie nur, wenn die Theorie tatsächlich diese Vorhersage trifft. Da in der Psychologie Theorien meistens nur verbal vorliegen und verbale Formulierungen oft zu unpräzise oder uneindeutig sind, um exakte Vorhersagen aus der Theorie abzuleiten, ist die empirische Überprüfung von Theorien in vielen Fällen nicht aussagekräftig \autocite{van_dongen_productive_2025}.

In diesem Kontext sind die Untersuchungen von \textcite{brotzeller_exploring_2025} zu sehen, die der Erforschung von Selbstkonzeptänderungen nach diskrepantem Feedback dienen. Genauer gesagt untersuchen \Citeauthor{brotzeller_exploring_2025}, welchen Einfluss Eigenschaften von diskrepantem Feedback auf nachfolgende Änderungen des Selbstkonzepts nehmen. Da die bisherige Forschung inkonsistente Befunde bezüglich des Einflusses einer dieser Eigenschaften produziert hat, skizzieren die Autoren eine theoretische Erklärung, welche die Systematik hinter dieser Inkonsistenz aufdecken soll, und überprüfen sie anhand einer Vorhersage, die aus ihr folgt.

Da diese Erklärung, fortan Theorie genannt, nur verbal vorliegt, dient die vorliegende Arbeit der Überprüfung, ob die Autoren ihre theoretischen Annahmen präzise und kohärent genug formuliert haben, um eine exakte Vorhersage treffen zu können. Zu diesem Zweck formalisiert die Arbeit im Teil \enquote{Untersuchung der theoretischen Fundierung} vor dem Hintergrund des Productive"=Explanation"=Framework \autocite{van_dongen_productive_2025} die theoretischen Annahmen der Autoren als Grundlage für ein mögliches formales Modell. Dabei stellt sich unter anderem heraus, dass der Erklärungswert der Theorie der Autoren, selbst wenn sich ihre Vorhersagen bestätigen sollten, als eher gering einzuschätzen ist. Der anschließende Teil \enquote{Reproduktion von Studie 4} dient der Reproduktion der Ergebnisse, mit deren Hilfe \Citeauthor{brotzeller_exploring_2025} die Gültigkeit ihrer theoretischen Erklärung überprüfen wollen. Zwar lassen sich die Ergebnisse reproduzieren, was für die Schlussfolgerung der Autoren spricht, dass die Theorie nicht gültig ist. Allerdings können Aspekte der Theorie diskutiert werden, aufgrund deren sich Zweifel an dieser Schlussfolgerung anmelden lassen. Mit dieser Diskussion beginnt der abschließende Teil der vorliegenden Arbeit. In seinem weiteren Verlauf wird erörtert, wie sich die Erklärungskraft der Theorie außerhalb des Productive"=Explanation"=Framework überprüfen und bewerten lässt.

\label{sec:intro}


    \section{Untersuchung der theoretischen Fundierung}
\label{sec:search}

\subsection{Grundannahmen}
\label{sec:search-base}

\subsection{Ein allgemeines Erklärungsgesetz}
\label{sec:search-rule}

\subsection{Anschlussfragen}
\label{sec:search-questions}

\subsection{Ein subjektiver Schwellenwert für die Wahrnehmung von Verbesserungsmöglichkeiten}
\label{sec:search-thresh}

\subsection{Die Evaluationsmotive als logische Folgen niedriger und hoher Verbesserungsmöglichkeiten}
\label{sec:search-eval}

\subsection{Der Positivbias und der Negativbias als logische Folgen der Evaluationsmotive}
\label{sec:search-bias}

    \section{Reproduktion von Studie 4}
\label{sec:repro}

\subsection{Umfang der Reproduktion und Weiterverarbeitung der Primärdaten}
\label{sec:repro-horizon}

\subsection{Implementierung und Durchführung der Berechnung}
\label{sec:repro-implementation}

\subsection{Vergleich der Ergebnisse}
\label{sec:repro-comparison}

\subsection{Neubewertung der Schlussfolgerungen}
\label{sec:repro-evaluation}
    \section{Diskussion}
\label{sec:discussion}

Dieses Kapitel enthält weitere Überlegungen und Anmerkungen zur Güte und empirischen Überprüfung der Theorie von \textcite{brotzeller_exploring_2025}. Zunächst wird eine Implikation der Theorie offengelegt, die daran zweifeln lässt, dass Studie 4 tatsächlich die Existenz des Phänomens und damit indirekt die Theorie widerlegen konnte. Von dem hypothetischen Szenario ausgehend, dass das Phänomen beobachtbar ist, wird anschließend argumentiert, dass in diesem Fall nur eine direkte Messung von Self"=Improvement und Self"=Enhancement den Nachweis dafür erbringen kann, dass die Theorie eine treffende Beschreibung der Realität ist. Abschließend werden die positiven Merkmale der Theorie herausgestellt, die erkennbar werden, wenn man das Productive"=Explanation"=Framework \autocite{van_dongen_productive_2025} verlässt und sich allgemeinen Gütekriterien für Theorien zuwendet.

\subsection{Von kontinuierlichen zu dichotomen Wahrnehmungen der Verbesserungsmöglichkeiten}
\label{sec:discussion-alternatives}

In der Theorie der Autoren sollte eine Motivation zu Self"=Improvement ausgelöst werden, sobald Personen Verbesserungsmöglichkeiten wahrnehmen; eine Motivation zu Self"=Enhancement hingegen, wenn sie keine solchen Möglichkeiten wahrnehmen. Das bedeutet, dass die Wahrnehmung von Verbesserungsmöglichkeiten eine dichotome Variable ist: Entweder nehmen Versuchspersonen Möglichkeiten wahr (1) oder nicht (0) (vgl. \cref{fig:abb-4}).

Im vorherigen Kapitel wurde dargelegt, dass \textcite{brotzeller_exploring_2025} die Gültigkeit der Theorie anzweifeln, weil sie in Studie 4 das Phänomen nicht identifizieren konnten, das aus ihr folgen würde (vgl. \cref{fig:placeholder-3}). Es wurde argumentiert, dass diese Schlussfolgerung nur so lange gültig ist, wie Studie 4 unzweifelhaft zeigen konnte, dass das Phänomen tatsächlich nicht existiert. Anlass zum Zweifel bietet der Verdacht, dass die experimentelle Manipulation der dichotomen Wahrnehmung von Verbesserungsmöglichkeiten nicht erfolgreich war.

Die Autoren gehen offensichtlich davon aus, dass die Wahrnehmung von Verbesserungsmöglichkeiten in der Realität zunächst eine kontinuierliche Variable ist. Das bedeutet, dass Versuchspersonen mehr oder weniger Möglichkeiten zur Verbesserung wahrnehmen. Die Manipulation in Studie 4 zielte darauf ab, die kontinuierliche Wahrnehmung der Versuchspersonen so zu beeinflussen, dass eine Experimentalgruppe (Gruppe 1) hohe, die andere Gruppe (Gruppe 2) hingegen niedrige Verbesserungsmöglichkeiten  wahrnimmt. Darüber hinaus erhoben \textcite{brotzeller_exploring_2025} am Ende von Studie 4, wie viele Möglichkeiten zur Verbesserung die Teilnehmer beider Gruppen tatsächlich wahrnahmen, und zwar nicht als dichotome, sondern als kontinuierliche Variable auf einer Likert"=Skala von 1--9.

Wenn sich Versuchspersonen eine kontinuierliche Wahrnehmung davon bilden, wie viele Möglichkeiten zur Verbesserung der Untersuchungskontext bereitstellt –- Wieso setzt die Theorie dann eine dichotome Wahrnehmung voraus? Die Antwort lautet: Die Theorie spezifiziert lediglich die Wertebereiche auf der Skala der kontinuierlichen Wahrnehmung, innerhalb derer die Bedingungen gegeben sind, die Self"=Improvement oder Self"=Enhancement auslösen. Da die Theorie mutmaßlich nicht vorsieht, dass es einen Wertebereich auf der Skala gibt, in dem weder die Bedingungen für das eine noch für das andere Motiv gegeben sind, grenzt der Wertebereich für Self"=Improvement an den Wertebereich für Self"=Enhancement an. Es muss also eine Funktion geben, die für den konkreten Wert an Verbesserungsmöglichkeiten, den eine Person wahrnimmt, bestimmt, ob dieser noch im Wertebereich für Self"=Enhancement (0) oder bereits im Wertebereich für Self"=Improvement liegt (1). 

Wie bereits dargelegt gehen \textcite{brotzeller_exploring_2025} davon aus, dass Personen eine Motivation zu Self"=Improvement entwickeln, sobald es ihnen möglich erscheint, sich zu verbessern. Der Schwellenwert zwischen dem Wertebereich für Self"=Improvement und Self"=Enhancement kann vor diesem Hintergrund als das Mindestmaß an Verbesserungsmöglichkeiten interpretiert werden, das Versuchspersonen wahrnehmen müssen, damit ihnen Verbesserung möglich erscheint.

Der Erfolg der Manipulation in Studie 4 bemisst sich daran, ob in Gruppe 1 Werte für die wahrgenommenen Verbesserungsmöglichkeiten, die im Wertebereich für Self"=Improvement liegen, und in Gruppe 2 Werte aus dem Wertebereich für Self"=Enhancement induziert werden konnten. War die Manipulation in diesem Sinne nicht erfolgreich, hätte das Phänomen gar nicht auftreten können. Gibt es also berechtigte Zweifel am Erfolg der Manipulation, kann aus dem Nicht-Auftreten des Phänomens in Studie 4 nicht automatisch auf seine Non"=Existenz geschlossen werden.

Der gerichtete t"=Test für Mittelwertsunterschiede, mit dem \textcite{brotzeller_exploring_2025} den Erfolg der Manipulation in Studie 4 überprüfen, kann keine Aussage darüber treffen, ob die Werte der kontinuierlichen Wahrnehmung von Verbesserungsmöglichkeiten in den beiden experimentellen Gruppen tatsächlich im Wertebereich des jeweiligen Motivs lagen. Er weist lediglich nach, dass die Versuchspersonen in Gruppe 1 im Durchschnitt mehr Verbesserungsmöglichkeiten wahrnahmen als Personen in Gruppe 2. Das bedeutet aber noch nicht, dass die Werte der Wahrnehmungen in Gruppe 1 im Schnitt über dem Schwellenwert für Self"=Improvement lagen und die Werte in Gruppe 2 im Schnitt unter dem Schwellenwert. Das signifikante Ergebnis des t"=Tests belegt also nicht, dass das Phänomen hätte auftreten müssen.

\Textcite{brotzeller_exploring_2025} weisen bei der Diskussion der Limitationen der durchgeführten Studien auf die Möglichkeit hin, dass die Wahrnehmung von niedrigen Verbesserungsmöglichkeiten noch über dem Schwellenwert liegt, ab dem Self"=Improvement möglich erscheint. Sie spekulieren, ob es in Studie 4 vielleicht nötig gewesen wäre, bei Personen in Gruppe 2 die Wahrnehmung zu induzieren, dass es überhaupt keine (statt geringer) Verbesserungsmöglichkeiten gibt, um den Schwellenwert zu unterschreiten und damit Self"=Enhancement zu provozieren. Wäre dem tatsächlich so, könnte man erklären, warum in Studie 4 nicht nur die Personen aus Gruppe 1, sondern auch die Personen aus Gruppe 2 einen Negativbias in den Änderungen ihrer Selbstkonzeptänderungen zeigten. 

Dass \textcite{brotzeller_exploring_2025} über diese Möglichkeit nachdenken, deutet darauf hin, dass sich aus ihrer Theorie keine Erwartung über die genaue Lage des Schwellenwerts für Self"=Improvement ableiten lässt. Die Beantwortung der Fragestellung, ob es das Phänomen gibt oder nicht, setzt allerdings auch nicht voraus, dass man die genaue Lage des Schwellenwerts kennt. Auch wenn man sich nicht sicher sein kann, dass die Wahrnehmung geringer Verbesserungsmöglichkeiten unter dem Schwellenwert für Self"=Improvement liegt, gibt es doch einen Wert, der in jedem Fall unter dem Schwellenwert liegen muss: Der Minimalwert der Skala, d.h. die Wahrnehmung, dass es überhaupt keine Verbesserungsmöglichkeiten gibt. Würde man messen, dass Personen in Gruppe 2 auch dann noch einen Negativbias zeigen, wenn sie überhaupt keine Verbesserungsmöglichkeiten wahrnehmen, hieße das, dass es keinen Wert gibt, bei dem sie sich für Self-Enhancement statt für Self-Improvement entscheiden würden. 

\subsection{Die Notwendigkeit der direkten Messung der Evaluationsmotive}
\label{sec:discussion-measure}

Im vorangegangenen Abschnitt wurden Überlegungen darüber angestellt, wie die Existenz des Phänomens nachgewiesen oder widerlegt werden kann. Grundlage dafür war die Überlegung, dass die Theorie nur dann ein Abbild der Realität sein kann, wenn das Phänomen, das aus ihr folgt, tatsächlich existiert. Nun soll das Szenario beleuchtet werden, in dem das Phänomen beobachtbar ist -– Wäre die Theorie dann automatisch ein Abbild der Realität?

Nein, denn es könnten auch andere Prozesse als Self"=Improvement und Self"=Enhancement zwischen den Wahrnehmungen von Verbesserungsmöglichkeiten und der Richtung der Asymmetrie in den Selbstkonzeptänderungen von Versuchspersonen vermitteln. Es bedarf also weiterer Analysen, die nachweisen können, dass diese beiden Motive für das Auftreten des Negativ- bzw. Positivbias verantwortlich sind. Solche Analysen setzen die direkte Messung der Motive in beiden experimentellen Gruppen voraus. Man müsste zeigen können, dass die meisten Personen in Gruppe 1 zu Self"=Improvement und die meisten Personen in Gruppe 2 zu Self"=Enhancement motiviert sind.

\subsection{Bewertung der Theorie außerhalb des Productive"=Explanation"=Framework}
\label{sec:discussion-relations}

Im ersten Kapitel der vorliegenden Arbeit wurde die Güte der Theorie unter dem Richtmaß einer bestimmten Auffassung davon bewertet, was eine gute erklärende Theorie ausmacht. Das Kapitel endete mit dem Fazit, dass die Theorie nach der Definition des Productive"=Explanation"=Frameworks eine geringe Erklärungskraft besitzt \autocite{van_dongen_productive_2025}. Es besteht allerdings die Möglichkeit, dass man zu einem anderen Schluss käme, wenn man eine andere Definition von guten Erklärungen anlegen würde.

Um der Tatsache Rechnung zu tragen, dass kein allgemeiner Konsens darüber herrscht, welche Merkmale eine gute Theorie ausmachen, wendet sich dieser letzte Abschnitt der Diskussion den Merkmalen zu, die jede Theorie, unabhängig von einer spezifischen Definition guter Erklärungen, mindestens aufweisen muss, um überhaupt als Theorie gelten zu dürfen.

Laut \textcite{sutton_what_1995} ist es viel einfacher zu definieren, was eine Theorie nicht ist, als was eine gute Theorie ausmacht. Sie erstellen deshalb eine Liste von Merkmalen, die Veröffentlichungen aufweisen, welche zwar den Anspruch haben, eine Theorie zu präsentieren, diesem Anspruch aber eigentlich nicht gerecht werden. Weist ein Artikel diese Merkmale nicht auf, spricht das dafür, dass er zumindest eine Theorie enthält, die als solche gelten kann. Das ist die grundlegende Voraussetzung für alle weiteren Gedanken über die Erklärungskraft der Theorie.

\Textcite{sutton_what_1995} begreifen es als systematisches Problem, dass viele Artikel die angesprochenen Merkmale aufweisen; sie seien das Ergebnis von Versuchen der Autoren solcher Artikel, die Aufgabe der Theorienentwicklung so weit im Aufwand und im Anforderungsniveau zu begrenzen, dass sie mit den an sich schon sehr ressourcenaufwändigen und rigiden Arbeitsabläufen des empirischen Hypothesentestens vereinbar bleibt. Da die psychologische Fachwelt auf die Replikationskrise in erster Linie damit reagierte, die Standards für saubere Methodik zu erhöhen \autocite{van_dongen_productive_2025}, ist es seitdem vielleicht sogar noch schwerer geworden, saubere Theorienentwicklung und saubere empirische Methodik unter einen Hut zu bringen. Vor diesem Hintergrund muss zunächst positiv hervorgehoben werden, dass \textcite{brotzeller_exploring_2025} überhaupt ein theoretisches Argument präsentieren und in Studie 4 überprüfen. Insbesondere, da sie offensichtlich auf keine der von \textcite{sutton_what_1995} identifizierten Kompromissstrategien zurückgegriffen haben.

\Textcite{brotzeller_exploring_2025} erwarten in Studie 4 einen Interaktionseffekt zwischen den wahrgenommenen Verbesserungsmöglichkeiten und der Richtung der Diskrepanz. Laut \textcite{sutton_what_1995} würden viele Autoren vor dem Hintergrund dieser Hypothese Unterschiede in den wahrgenommenen Verbesserungsmöglichkeiten als ausreichende Erklärung für die inkonsistenten Befunde bezüglich der Richtung der Asymmetrie in den Selbstkonzeptänderungen missverstehen. \Citeauthor{brotzeller_exploring_2025} bemühen sich jedoch richtigerweise um eine Erklärung, warum diese Zusammenhänge zu erwarten sind.

\Textcite{brotzeller_exploring_2025} verwenden einen bestehenden Erklärungsansatz aus der Literatur wieder. Anstatt nur auf ihn zu verweisen, führen sie ihn jedoch aus und leiten begründet eine Hypothese aus ihm ab. \Textcite{sutton_what_1995} beobachteten häufig, dass Autoren eine Theorie entwickeln, indem sie empirische Befunde ohne weitere inhaltliche Begründung zueinander in Beziehung setzen. Demgegenüber zitieren \textcite{brotzeller_exploring_2025} empirische Befunde stets zur Fundierung einer inhaltlichen Begründung für den Zusammenhang zwischen Variablen (Beispiel: Die Wahrnehmung geringer Verbesserungsmöglichkeiten provoziert Self"=Enhancement, weil das Annehmen von negativem Feedback unter diesen Umständen zu permanenten negativen Selbstwahrnehmungen führen würde).

\Textcite{sutton_what_1995} zufolge lassen sich gute Theorien nicht nur verbal, sondern auch in Form eines Diagramms darstellen. Das setzt natürlich voraus, dass die theoretischen Annahmen präzise genug formuliert sind, um sie in einem Diagramm darstellen zu können. \Cref{fig:abb-4} belegt, dass die Theorie der Autoren dieses Kriterium erfüllt. 

Ausgangspunkt für die Untersuchungen von \textcite{brotzeller_exploring_2025} ist, dass noch kein Faktor identifiziert wurde, der bestimmt, wann ein Negativbias und wann ein Positivbias auftritt. Ein mögliches Vorgehen wäre es gewesen, den Einfluss möglichst vieler Faktoren auf die Richtung der Asymmetrie zu messen, in der Hoffnung, dass einer dieser Faktoren einen signifikanten Einfluss ausüben würde. \Citeauthor{brotzeller_exploring_2025} gehen allerdings anders vor: Sie stellen erst theoretische Überlegungen darüber an, welche Prozesse einen Positivbias und einen Negativbias erklären würden, und leiten daraus einen in Frage kommenden Faktor ab. \textcite{sutton_what_1995} zufolge gibt es durchaus Stimmen in der Literatur, die sich für das erste Vorgehen aussprechen. Diesen Stimmen zufolge müsse man als Basis für die Entwicklung erfolgreicher Theorien zunächst ein breites empirisches Wissen über möglichst viele Einflussfaktoren und ihr Zusammenwirken aufbauen. Die theoriegeleitete Suche der Autoren nach einem Einflussfaktor ist angesichts der folgenden beiden Vorteile aber mindestens ebenso gerechtfertigt: Sie ist ressourcensparender und erprobt die Erklärungskraft bestehender Erklärungsprinzipien in einem neuen Anwendungsbereich.

    \section{Schlussbemerkungen}
\label{sec:end}

    % Literaturverzeichnis
    \printbibliography

    \appendix

    \section{Annahmentabelle}
\label{app:table}

\begin{landscape}   
    \begin{xltabular}{\linewidth}{l >{\RaggedRight\hsize=1.4\hsize}X >{\RaggedRight\hsize=0.9\hsize}X c >{\RaggedRight\hsize=0.7\hsize}X r}

    % --- Tabellenkopf (erste Seite) ---
    \caption{Theoretische Annahmen und Konzepte nach \autocite{brotzeller_exploring_2025}.\label{tab:annahmetabelle}} \\
    \toprule
    \textbf{Seite} & \textbf{Zitat} & \textbf{Annahme} & \textbf{Rel.} & \textbf{Konzepte} & \textbf{ID} \\
    \midrule
    \endfirsthead

    % --- Tabellenkopf (Folgeseiten) ---
    \caption[]{Theoretische Annahmen (Fortsetzung)} \\
    \toprule
    \textbf{Seite} & \textbf{Zitat} & \textbf{Annahme} & \textbf{Rel.} & \textbf{Konzepte} & \textbf{ID} \\
    \midrule
    \endhead

    % --- Tabellenfuß (Folgeseiten) ---
    \midrule
    \multicolumn{6}{r}{\textit{Fortsetzung auf der nächsten Seite...}} \\
    \endfoot

    % --- Tabellenfuß (letzte Seite) ---
    \bottomrule
    \multicolumn{6}{p{\linewidth}}{%
        \footnotesize \textit{Hinweis.} Alle Zitate stammen aus \textcite{brotzeller_exploring_2025}. Aus jedem Zitat wurden die einzelnen Annahmen extrahiert, die sie enthält. Die Annahmen 1--18 betreffen Definitionen von Phänomenen und Variablen, die Annahmen 19--45 hingegen die Annahmen der von den Autoren dargestellten Theorie. Jeder Annahme werden die Konzepte zugeordnet, die sie in Beziehung setzt. Ein Higher"=Order"=Concept (HOC) umschließt die darauffolgenden Konzepte. Legende Rel.: n = Naming, i = Conceptual Implication, c = Causation, p = Prediction
    } \\
    \endlastfoot

    % --- INHALT ---
    
    1731 & \multirow{3}{=}{In both settings, self-relevant feedback can shape people’s self-concept, defined as a person’s perception of themselves (Bem, 1972; Shavelson et al., 1976). However, the extent to which discrepant external feedback leads to changes in such self-perceptions (i.e., self-concept change) varies considerably: In some cases, receiving external feedback leads to self-concept change in accordance with the feedback, while, in other cases, even highly discrepant feedback does not lead to self-concept change. [\dots] Although a considerable amount of research has examined self-concept change after self-relevant feedback, many unanswered questions remain.}
         & Finding: Self-concept change after self-relevant feedback 
         & n & SELF-CONCEPT CHANGE AFTER SELF-RELEVANT FEEDBACK & \hypertarget{id1}{1} \\
    % \addlinespace
         & % Zitat wird nicht wiederholt
         & Finding: Discrepant external feedback leads to changes in self-perceptions (i.e., self-concept change) 
         & c & HOC: SELF-CONCEPT CHANGE AFTER SELF-RELEVANT FEEDBACK, DISCREPANT FEEDBACK, SELF-CONCEPT CHANGE & \hypertarget{id2}{2} \\
    % \addlinespace
         & 
         & Finding: Self-concept change in accordance with the feedback 
         & n & SELF-CONCEPT CHANGE AFTER SELF-RELEVANT FEEDBACK & \hypertarget{id3}{3} \\
    \midrule

    1731 & \multirow{2}{=}{In the present research, we focus on the latter of the three features [i.e., features of the feedback itself] and examine the size of the discrepancy between one’s self-view and the feedback one receives as well as the direction of the discrepancy -- that is, whether the feedback is positive and suggests an upward adjustment of one’s self-concept (e.g., “I obviously cook better than I thought I would”) or whether it is negative and suggests a downward adjustment (e.g., “I obviously cook worse than I thought I would”).}
         & Size of discrepancy between one’s self-view and the feedback one receives
         & n & SIZE OF DISCREPANCY & \hypertarget{id4}{4} \\
    \addlinespace
         & 
         & Direction of the discrepancy between one’s self-view and the feedback one receives
         & n & DIRECTION OF DISCREPANCY & \hypertarget{id5}{5} \\\\\\\\\\\\
    \midrule

    1732 & \multirow{2}{=}{Studies investigating the effect of the size of the discrepancy between self-concept and feedback consistently demonstrate that larger discrepancies lead to more self-concept change (i.e., larger differences between previous and current self-perceptions) than smaller discrepancies, except for extreme and likely implausible discrepancies (Bergin, 1962; Binderman et al., 1972; Kube et al., 2022).}
         & Finding: Larger discrepancies between self-concept and feedback lead to more self-concept change 
         & p, c & SIZE OF DISCREPANCY, SELF-CONCEPT CHANGE & \hypertarget{id6}{6} \\
    \addlinespace
         & 
         & Finding: Extreme and likely implausible discrepancies do not lead to self-concept change 
         & p 0 & SIZE OF DISCREPANCY, SELF-CONCEPT CHANGE & \hypertarget{id7}{7} \\\\\\
    \midrule

    1732 & \multirow{3}{=}{Regarding the direction of the discrepancy, however, the empirical findings are less conclusive. Several studies show that positive and negative feedback lead to different amounts of self-concept change. [\dots] 
    % Interestingly, it is unclear which of the two types of feedback leads to larger self-concept change: 
    The majority of studies find larger self-concept change after positive than after negative feedback, indicating a positivity bias in the processing of self-relevant information (Eil \& Rao, 2011; Elder et al., 2022; Korn et al., 2012; Möbius et al., 2022). [\dots]
    %The term positivity bias hereby is not meant to imply that such processing of self-relevant information is irrational; we merely use it to describe cases in which positive feedback produces more self-concept change than negative feedback. 
    Notably, two recent studies demonstrate larger self-concept change after negative than after positive feedback (Ertac, 2011; Müller-Pinzler et al., 2019) -- a pattern that rather suggests a negativity bias.}
         & Positive and negative feedback lead to different amounts of self-concept change 
         & p & HOC: ASYMMETRIC SELF-CONCEPT CHANGE, DIRECTION OF DISCREPANCY, SELF-CONCEPT CHANGE & \hypertarget{id8}{8} \\
    \addlinespace
         & 
         & Positivity bias in self"=concept change
         & n & POSITIVITY BIAS & \hypertarget{id9}{9} \\
    \addlinespace
         & 
         & Positive feedback produces more self-concept change than negative feedback 
         & p & HOC: POSITIVITY BIAS, DIRECTION OF DISCREPANCY, SELF-CONCEPT CHANGE & \hypertarget{id10}{10} \\\\\\
    \addlinespace
         & 
         & Larger self"=concept change after negative than after positive feedback 
         & p & HOC: NEGATIVITY BIAS, DIRECTION OF DISCREPANCY, SELF-CONCEPT CHANGE & \hypertarget{id11}{11} \\
    \addlinespace
         & 
         & Negativity bias in self"=concept change
         & n & NEGATIVITY BIAS & \hypertarget{id12}{12} \\
    \midrule
    1732 & 1732	While we are mainly interested in the main effects of size and direction of discrepancy separately, we also explore whether they interact in producing self-concept change. Prior studies have largely neglected the possible interaction effects of these variables. & The size and direction of discrepancy might interact in producing self-concept change & p ? & SIZE OF DISCREPANCY, DIRECTION OF DISCREPANCY,
SELF"=CONCEPT CHANGE & \hypertarget{id13}{13} \\\\
    \midrule


    1732 & \multirow{4}{=}{Our own definition of self-concept change builds upon the definition by Shavelson et al. (1976) and attempts to be even more precise: We argue that self-concept change has occurred whenever a person’s perception of themselves on a specific self-relevant dimension at a given time point differs from a previous self-perception on the same dimension.}
         & Self-concept change 
         & n & SELF-CONCEPT CHANGE & \hypertarget{id14}{14} \\
    \addlinespace
         & 
         & Self-perception on a specific self-relevant dimension at time point t1 
         & n & SELF-PERCEPTION & \hypertarget{id15}{15} \\
    \addlinespace
         & 
         & Self-perception on a specific self-relevant dimension at time point t2 
         & n & SELF-PERCEPTION & \hypertarget{id16}{16} \\
    \addlinespace
         & 
         & Self-concept change has occurred when $\mathrm{self-perception}_{\mathrm{t1}}$ differs from $\mathrm{self-perception}_{\mathrm{t2}}$
         & i & SELF-PERCEPTION, DIFFERENCE IN SELF-PERCEPTIONS, SELF-CONCEPT CHANGE & \hypertarget{id17}{17} \\
    \midrule

    1732 & Feedback means any kind of external information that a person receives on a self-relevant dimension (e.g., on a trait or an ability). Importantly, this feedback must be perceived as diagnostically relevant for this specific dimension: [\dots]
         & External, self-relevant feedback 
         & n & DISCREPANT FEEDBACK & \hypertarget{id18}{18} \\
    \midrule

    1733 & \multirow{2}{=}{There are different theoretical approaches to explaining the positivity and negativity bias in self-concept change and the contradictory findings that have resulted from previous research. One such approach focuses on two processes that shape how people perceive and integrate feedback into the self-concept: self-enhancement and self-improvement.}
         & Perception of feedback 
         & n & PERCEPTION OF FEEDBACK & \hypertarget{id19}{19} \\
    \addlinespace
         & 
         & Self-enhancement processes shape how people perceive and integrate feedback into the self-concept 
         & c & SELF-ENHANCEMENT PROCESSES, PERCEPTION OF FEEDBACK, SELF-CONCEPT CHANGE & \hypertarget{id20}{20} \\\\
    \addlinespace
         & 
         & Self-improvement processes shape how people perceive and integrate feedback into the self-concept 
         & c & SELF-IMPROVEMENT PROCESSES, PERCEPTION OF FEEDBACK, SELF-CONCEPT CHANGE & \hypertarget{id21}{21} \\
    \midrule

    1733 & \multirow{2}{=}{While self-enhancement describes biases in processing and interpreting information in a self-serving fashion (Heine \& Hamamura, 2007), self-improvement describes biases aimed at reducing discrepancies between an “is-state” and a desirable “ought-state” (Kurmann, 2006). }
         & Self-enhancement processes are biases in processing and interpreting information in a self-serving fashion 
         & n & SELF-ENHANCEMENT PROCESSES & \hypertarget{id22}{22} \\
    \addlinespace
         & 
         & Self-Improvement processes are biases aimed at approaching a desirable ought-state 
         & n & SELF-IMPROVEMENT PROCESSES & \hypertarget{id23}{23} \\\\
    \midrule

    1733 & \multirow{3}{=}{Both self-enhancement and self-improvement assume that people are motivated to maintain a positive view of themselves even (or particularly) in the face of disconfirming feedback (Taylor \& Brown, 1988).}
         & Motivation to maintain a positive view of oneself 
         & n & POSITIVE SELF MOTIVE & \hypertarget{id24}{24} \\
    \addlinespace
         & 
         & Self-enhancement assumes a motivation to maintain a positive view of oneself 
         & i & SELF-ENHANCEMENT MOTIVE, POSITIVE SELF MOTIVE & \hypertarget{id25}{25} \\
    \addlinespace
         & 
         & Self-improvement assumes a motivation to maintain a positive view of oneself 
         & i & SELF-IMPROVEMENT MOTIVE, POSITIVE SELF MOTIVE & \hypertarget{id26}{26} \\\\\\\\
    \midrule

    1733 & \multirow{3}{=}{When a person receiving feedback is motivated to self-enhance, they should focus on positive and dismiss negative information as the latter is perceived as threatening one’s positive self-view. Therefore, a self-enhancement motive should lead to positively biased self-concept change.}
         & Motivation to self-enhance 
         & n & SELF-ENHANCEMENT MOTIVE & \hypertarget{id27}{27}\\
    \addlinespace
         & 
         & When a person is motivated to self-enhance, negative discrepant feedback is perceived as threatening one’s positive self-view 
         & i & SELF-ENHANCEMENT MOTIVE, HOC: PERCEPTION OF FEEDBACK, THREAD BY NEGATIVE FEEDBACK & \hypertarget{id28}{28} \\
    \addlinespace
         & 
         & People should focus on positive and dismiss negative information as the latter is perceived as threatening one’s self-view 
         & c & THREAD BY NEGATIVE FEEDBACK, FOCUS ON POSITIVE FEEDBACK & \hypertarget{id29}{29} \\
    \addlinespace
         & 
         & Because people focus on positive information as negative information is perceived as threatening under a self-enhancement motive, a self-enhancement motive should trigger a positivity bias in self-concept change
         & c & SELF-ENHANCEMENT MOTIVE, THREAD BY NEGATIVE FEEDBACK, FOCUS ON POSITIVE FEEDBACK, POSITIVITY BIAS & \hypertarget{id30}{30} \\
    \midrule

    1733 & When a person receiving feedback is motivated to self-improve, however, negative feedback is more informative than positive feedback because the former highlights opportunities for improvement. In other words, such a person should be negatively biased in changing their self-concept.
         & Motivation to self-improve 
         & n & SELF-IMPROVEMENT MOTIVE & \hypertarget{id31}{31} \\
    \addlinespace
         & 
         & If a person is motivated to self-improve, negative discrepant feedback is more informative than positive 
         & i & SELF-IMPROVEMENT MOTIVE, HOC: PERCEPTION OF FEEDBACK, INFORMATIONAL VALUE OF [NEGATIVE|POSITIVE] FEEDBACK & \hypertarget{id32}{32} \\
    \addlinespace
         & 
         & The perception that negative feedback is more informative (for improvement) under a self-improvement motive should lead to a negativity bias in self-concept change
         & c, i & SELF-IMPROVEMENT MOTIVE, INFORMATIVITY OF FEEDBACK, NEGATIVITY BIAS & \hypertarget{id33}{33} \\
    \midrule

    1733 & \multirow{3}{=}{
    While self-improvement is triggered in particular when a person perceives that they can overcome is-ought discrepancies (e.g., by practicing or rehearsing), 
    self-enhancement should be triggered when a person perceives it as impossible to improve on the self-concept aspect in question (Müller-Pinzler et al., 2019). 
    When the aspect of the self-concept is perceived as fixed and unimprovable, negative feedback does not have an informational value toward improving oneself [\dots].
    but is, instead, particularly threatening to one’s positive self-view (Dunning, 1995; Dweck et al., 1995; Levy \& Dweck, 1998). 
    In such cases, the only possibility of maintaining one’s positive 
    % self-view is to self-enhance. Perceiving little opportunity for improvement should therefore trigger self-enhancement and produce positively biased self-concept change.
    }
         & Perceived possibility to improve on the self-concept aspect in question 
         & n & OPPORTUNITY FOR IMPROVEMENT & \hypertarget{id34}{34} \\
    \addlinespace
         & 
         & Perceiving a possibility to improve, i.e. a possibility to overcome is-ought discrepancies, triggers a self-improvement motivation 
         & c & OPPORTUNITY FOR IMPROVEMENT, SELF-IMPROVEMENT MOTIVE & \hypertarget{id35}{35} \\
    \addlinespace
         & 
         & If people perceive it as impossible to improve, negative feedback has no informational value 
         & i -1 & OPPORTUNITY FOR IMPROVEMENT, INFORMATIONAL VALUE OF NEGATIVE FEEDBACK & \hypertarget{id36}{36} \\\\
         
    \addlinespace
         & \multirow{2}{=}{
         self-view is to self-enhance. Perceiving little opportunity for improvement should therefore trigger self-enhancement and produce positively biased self-concept change.
         }
         & If people perceive it as impossible to improve, negative feedback is particularly threatening to one’s self-view 
         & i & OPPORTUNITY FOR IMPROVEMENT, THREAD BY NEGATIVE FEEDBACK & \hypertarget{id37}{37} \\
    \addlinespace
         & 
         & If negative feedback has no informational value but is perceived as threatening instead, self-enhancement is the only possibility of maintaining one’s positive self-view. Therefore, perceiving little opportunity for improvement triggers self-enhancement.
         & c, i & OPPORTUNITY FOR IMPROVEMENT, INFORMATIONAL VALUE OF NEGATIVE FEEDBACK, THREAD BY NEGATIVE FEEDBACK, SELF-ENHANCEMENT MOTIVE & \hypertarget{id38}{38} \\
    \midrule

    1733 & \multirow{4}{=}{Supporting this theorizing, a positivity bias—reflecting a self-enhancement process—has been empirically demonstrated on those self-concept aspects that are most likely to be perceived as fixed and unchangeable by most people (e.g., intelligence or beauty, see Eil \& Rao, 2011; Möbius et al., 2022) or if the study was designed such that participants likely saw little opportunity for improvement (e.g., one-shot feedback from third parties; see Elder et al., 2022; Korn et al., 2012). These findings are also consistent with other studies on belief updating after feedback (Lefebvre et al., 2017).}
         & Modifiability of the dimension of the self-concept in question 
         & n & MALLEABILITY & \hypertarget{id39}{39} \\
    \addlinespace
         & 
         & Perceiving that the self-concept aspect in question is likely to be modifiable promotes the perception that it is possible to improve on this aspect 
         & i & MALLEABILITY, OPPORTUNITY FOR IMPROVEMENT & \hypertarget{id40}{40} \\
    \addlinespace
         & 
         & Design of an empirical study 
         & n & STUDY DESIGN & \hypertarget{id41}{41} \\
    \addlinespace
         & 
         & Certain features of the study design promote or diminish the perceived opportunity for improvement 
         & i & STUDY DESIGN, OPPORTUNITY FOR IMPROVEMENT & \hypertarget{id42}{42} \\\\
    \midrule

    1733 & By contrast, when the self-concept aspect in question is perceived as improvable (“malleable”), negative feedback is more informative for self-improvement purposes than positive feedback (Strube, 2012). 
         & If the self-concept aspect in question is perceived as improvable, negative feedback is more informative for self-improvement purposes than positive feedback 
         & i & MALLEABILITY, INFORMATIONAL VALUE OF NEGATIVE FEEDBACK, INFORMATIONAL VALUE OF POSITIVE FEEDBACK & \hypertarget{id43}{43} \\\\\\\\\\\\\\\\
    \midrule
    1733 & \multirow{2}{=}{To sum up, the opportunity for improvement in conjunction with motives for self-enhancement and self-improvement may be a plausible explanation for the contradictory findings on self-concept change after negative vs. positive feedback. Yet, this explanation has not been systematically examined so far. Therefore, the present research investigates the role of the opportunity for improvement in asymmetric self-concept change.} & Contradictory findings on self-concept change after negative vs. positive feedback & n & CONTRADICTORY FINDINGS & \hypertarget{id44}{44}\\
    \addlinespace
    &
    & Asymmetric self-concept change & n & ASYMMETRIC SELF-CONCEPT CHANGE & \hypertarget{id45}{45}\\\\\\\\\\\\

\end{xltabular}
\end{landscape}
    \section{Reproduktion}
\label{app:b}

\noindent
\begin{minipage}{\textwidth}
    \begingroup
    \let\clearpage\relax
    \includepdf[pages=1, pagecommand={}, trim=0 0 0 5cm,
    clip]{reproduction-study4/re-implementation.pdf}
    \endgroup
\end{minipage}

\includepdf[pages={2-9},fitpaper]{reproduction-study4/re-implementation.pdf}
    \section{Erklärung zur Nutzung von generativer KI und KI-gestützten Technologien}

Bei der Erstellung dieser Arbeit habe ich, Benedikt Daniel Ehrenwirth, folgende/s Tool/s verwendet:
\begin{description}
    \item[Overleaf GPT AI Assistant] Verbesserung der sprachlichen Qualität und Lesbarkeit
    \item[DeepL] Verbesserung der sprachlichen Qualität und Lesbarkeit
\end{description}

Nach der Nutzung dieses Tools bzw. Dienstes habe ich den Inhalt überprüft, nach Bedarf bearbeitet und ich übernehme die volle Verantwortung für den Inhalt dieser Arbeit. Ich bestätige, dass diese Arbeit keine längeren Passagen (z.B. Zusammenfassung/Abstract der Arbeit, ganze Absätze im Text) an rein KI-generiertem Text enthält.

\vspace{2cm}

\noindent München, den 12.01.2026 \hfill\underline{\phantom{Benedikt Daniel Ehrenwirth}}\\
\hfill Benedikt Daniel Ehrenwirth
\end{document}