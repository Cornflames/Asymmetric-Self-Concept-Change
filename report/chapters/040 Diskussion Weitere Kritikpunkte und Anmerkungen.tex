\section{Diskussion}
\label{sec:discussion}

Dieses Kapitel enthält weitere Überlegungen und Anmerkungen zur Güte und empirischen Überprüfung der Theorie von \textcite{brotzeller_exploring_2025}. Zunächst wird eine Implikation der Theorie offengelegt, die daran zweifeln lässt, dass Studie 4 tatsächlich die Existenz des Phänomens und damit indirekt die Theorie widerlegen konnte. Von dem hypothetischen Szenario ausgehend, dass das Phänomen beobachtbar ist, wird anschließend argumentiert, dass in diesem Fall nur eine direkte Messung von Self"=Improvement und Self"=Enhancement den Nachweis dafür erbringen kann, dass die Theorie eine treffende Beschreibung der Realität ist. Abschließend werden die positiven Merkmale der Theorie herausgestellt, die erkennbar werden, wenn man das Productive"=Explanation"=Framework \autocite{van_dongen_productive_2025} verlässt und sich allgemeinen Gütekriterien für Theorien zuwendet.

\subsection{Von kontinuierlichen zu dichotomen Wahrnehmungen der Verbesserungsmöglichkeiten}
\label{sec:discussion-alternatives}

In der Theorie der Autoren sollte eine Motivation zu Self"=Improvement ausgelöst werden, sobald Personen Verbesserungsmöglichkeiten wahrnehmen; eine Motivation zu Self"=Enhancement hingegen, wenn sie keine solchen Möglichkeiten wahrnehmen. Das bedeutet, dass die Wahrnehmung von Verbesserungsmöglichkeiten eine dichotome Variable ist: Entweder nehmen Versuchspersonen Möglichkeiten wahr (1) oder nicht (0) (vgl. \cref{fig:abb-4}).

Im vorherigen Kapitel wurde dargelegt, dass \textcite{brotzeller_exploring_2025} die Gültigkeit der Theorie anzweifeln, weil sie in Studie 4 das Phänomen nicht identifizieren konnten, das aus ihr folgen würde (vgl. \cref{fig:placeholder-3}). Es wurde argumentiert, dass diese Schlussfolgerung nur so lange gültig ist, wie Studie 4 unzweifelhaft zeigen konnte, dass das Phänomen tatsächlich nicht existiert. Anlass zum Zweifel bietet der Verdacht, dass die experimentelle Manipulation der dichotomen Wahrnehmung von Verbesserungsmöglichkeiten nicht erfolgreich war.

Die Autoren gehen offensichtlich davon aus, dass die Wahrnehmung von Verbesserungsmöglichkeiten in der Realität zunächst eine kontinuierliche Variable ist. Das bedeutet, dass Versuchspersonen mehr oder weniger Möglichkeiten zur Verbesserung wahrnehmen. Die Manipulation in Studie 4 zielte darauf ab, die kontinuierliche Wahrnehmung der Versuchspersonen so zu beeinflussen, dass eine Experimentalgruppe (Gruppe 1) hohe, die andere Gruppe (Gruppe 2) hingegen niedrige Verbesserungsmöglichkeiten  wahrnimmt. Darüber hinaus erhoben \textcite{brotzeller_exploring_2025} am Ende von Studie 4, wie viele Möglichkeiten zur Verbesserung die Teilnehmer beider Gruppen tatsächlich wahrnahmen, und zwar nicht als dichotome, sondern als kontinuierliche Variable auf einer Likert"=Skala von 1--9.

Wenn sich Versuchspersonen eine kontinuierliche Wahrnehmung davon bilden, wie viele Möglichkeiten zur Verbesserung der Untersuchungskontext bereitstellt –- Wieso setzt die Theorie dann eine dichotome Wahrnehmung voraus? Die Antwort lautet: Die Theorie spezifiziert lediglich die Wertebereiche auf der Skala der kontinuierlichen Wahrnehmung, innerhalb derer die Bedingungen gegeben sind, die Self"=Improvement oder Self"=Enhancement auslösen. Da die Theorie mutmaßlich nicht vorsieht, dass es einen Wertebereich auf der Skala gibt, in dem weder die Bedingungen für das eine noch für das andere Motiv gegeben sind, grenzt der Wertebereich für Self"=Improvement an den Wertebereich für Self"=Enhancement an. Es muss also eine Funktion geben, die für den konkreten Wert an Verbesserungsmöglichkeiten, den eine Person wahrnimmt, bestimmt, ob dieser noch im Wertebereich für Self"=Enhancement (0) oder bereits im Wertebereich für Self"=Improvement liegt (1). 

Wie bereits dargelegt gehen \textcite{brotzeller_exploring_2025} davon aus, dass Personen eine Motivation zu Self"=Improvement entwickeln, sobald es ihnen möglich erscheint, sich zu verbessern. Der Schwellenwert zwischen dem Wertebereich für Self"=Improvement und Self"=Enhancement kann vor diesem Hintergrund als das Mindestmaß an Verbesserungsmöglichkeiten interpretiert werden, das Versuchspersonen wahrnehmen müssen, damit ihnen Verbesserung möglich erscheint.

Der Erfolg der Manipulation in Studie 4 bemisst sich daran, ob in Gruppe 1 Werte für die wahrgenommenen Verbesserungsmöglichkeiten, die im Wertebereich für Self"=Improvement liegen, und in Gruppe 2 Werte aus dem Wertebereich für Self"=Enhancement induziert werden konnten. War die Manipulation in diesem Sinne nicht erfolgreich, hätte das Phänomen gar nicht auftreten können. Gibt es also berechtigte Zweifel am Erfolg der Manipulation, kann aus dem Nicht-Auftreten des Phänomens in Studie 4 nicht automatisch auf seine Non"=Existenz geschlossen werden.

Der gerichtete t"=Test für Mittelwertsunterschiede, mit dem \textcite{brotzeller_exploring_2025} den Erfolg der Manipulation in Studie 4 überprüfen, kann keine Aussage darüber treffen, ob die Werte der kontinuierlichen Wahrnehmung von Verbesserungsmöglichkeiten in den beiden experimentellen Gruppen tatsächlich im Wertebereich des jeweiligen Motivs lagen. Er weist lediglich nach, dass die Versuchspersonen in Gruppe 1 im Durchschnitt mehr Verbesserungsmöglichkeiten wahrnahmen als Personen in Gruppe 2. Das bedeutet aber noch nicht, dass die Werte der Wahrnehmungen in Gruppe 1 im Schnitt über dem Schwellenwert für Self"=Improvement lagen und die Werte in Gruppe 2 im Schnitt unter dem Schwellenwert. Das signifikante Ergebnis des t"=Tests belegt also nicht, dass das Phänomen hätte auftreten müssen.

\Textcite{brotzeller_exploring_2025} weisen bei der Diskussion der Limitationen der durchgeführten Studien auf die Möglichkeit hin, dass die Wahrnehmung von niedrigen Verbesserungsmöglichkeiten noch über dem Schwellenwert liegt, ab dem Self"=Improvement möglich erscheint. Sie spekulieren, ob es in Studie 4 vielleicht nötig gewesen wäre, bei Personen in Gruppe 2 die Wahrnehmung zu induzieren, dass es überhaupt keine (statt geringer) Verbesserungsmöglichkeiten gibt, um den Schwellenwert zu unterschreiten und damit Self"=Enhancement zu provozieren. Wäre dem tatsächlich so, könnte man erklären, warum in Studie 4 nicht nur die Personen aus Gruppe 1, sondern auch die Personen aus Gruppe 2 einen Negativbias in den Änderungen ihrer Selbstkonzeptänderungen zeigten. 

Dass \textcite{brotzeller_exploring_2025} über diese Möglichkeit nachdenken, deutet darauf hin, dass sich aus ihrer Theorie keine Erwartung über die genaue Lage des Schwellenwerts für Self"=Improvement ableiten lässt. Die Beantwortung der Fragestellung, ob es das Phänomen gibt oder nicht, setzt allerdings auch nicht voraus, dass man die genaue Lage des Schwellenwerts kennt. Auch wenn man sich nicht sicher sein kann, dass die Wahrnehmung geringer Verbesserungsmöglichkeiten unter dem Schwellenwert für Self"=Improvement liegt, gibt es doch einen Wert, der in jedem Fall unter dem Schwellenwert liegen muss: Der Minimalwert der Skala, d.h. die Wahrnehmung, dass es überhaupt keine Verbesserungsmöglichkeiten gibt. Würde man messen, dass Personen in Gruppe 2 auch dann noch einen Negativbias zeigen, wenn sie überhaupt keine Verbesserungsmöglichkeiten wahrnehmen, hieße das, dass es keinen Wert gibt, bei dem sie sich für Self-Enhancement statt für Self-Improvement entscheiden würden. 

\subsection{Die Notwendigkeit der direkten Messung der Evaluationsmotive}
\label{sec:discussion-measure}

Im vorangegangenen Abschnitt wurden Überlegungen darüber angestellt, wie die Existenz des Phänomens nachgewiesen oder widerlegt werden kann. Grundlage dafür war die Überlegung, dass die Theorie nur dann ein Abbild der Realität sein kann, wenn das Phänomen, das aus ihr folgt, tatsächlich existiert. Nun soll das Szenario beleuchtet werden, in dem das Phänomen beobachtbar ist -– Wäre die Theorie dann automatisch ein Abbild der Realität?

Nein, denn es könnten auch andere Prozesse als Self"=Improvement und Self"=Enhancement zwischen den Wahrnehmungen von Verbesserungsmöglichkeiten und der Richtung der Asymmetrie in den Selbstkonzeptänderungen von Versuchspersonen vermitteln. Es bedarf also weiterer Analysen, die nachweisen können, dass diese beiden Motive für das Auftreten des Negativ- bzw. Positivbias verantwortlich sind. Solche Analysen setzen die direkte Messung der Motive in beiden experimentellen Gruppen voraus. Man müsste zeigen können, dass die meisten Personen in Gruppe 1 zu Self"=Improvement und die meisten Personen in Gruppe 2 zu Self"=Enhancement motiviert sind.

\subsection{Bewertung der Theorie außerhalb des Productive"=Explanation"=Framework}
\label{sec:discussion-relations}

Im ersten Kapitel der vorliegenden Arbeit wurde die Güte der Theorie unter dem Richtmaß einer bestimmten Auffassung davon bewertet, was eine gute erklärende Theorie ausmacht. Das Kapitel endete mit dem Fazit, dass die Theorie nach der Definition des Productive"=Explanation"=Frameworks eine geringe Erklärungskraft besitzt \autocite{van_dongen_productive_2025}. Es besteht allerdings die Möglichkeit, dass man zu einem anderen Schluss käme, wenn man eine andere Definition von guten Erklärungen anlegen würde.

Um der Tatsache Rechnung zu tragen, dass kein allgemeiner Konsens darüber herrscht, welche Merkmale eine gute Theorie ausmachen, wendet sich dieser letzte Abschnitt der Diskussion den Merkmalen zu, die jede Theorie, unabhängig von einer spezifischen Definition guter Erklärungen, mindestens aufweisen muss, um überhaupt als Theorie gelten zu dürfen.

Laut \textcite{sutton_what_1995} ist es viel einfacher zu definieren, was eine Theorie nicht ist, als was eine gute Theorie ausmacht. Sie erstellen deshalb eine Liste von Merkmalen, die Veröffentlichungen aufweisen, welche zwar den Anspruch haben, eine Theorie zu präsentieren, diesem Anspruch aber eigentlich nicht gerecht werden. Weist ein Artikel diese Merkmale nicht auf, spricht das dafür, dass er zumindest eine Theorie enthält, die als solche gelten kann. Das ist die grundlegende Voraussetzung für alle weiteren Gedanken über die Erklärungskraft der Theorie.

\Textcite{sutton_what_1995} begreifen es als systematisches Problem, dass viele Artikel die angesprochenen Merkmale aufweisen; sie seien das Ergebnis von Versuchen der Autoren solcher Artikel, die Aufgabe der Theorienentwicklung so weit im Aufwand und im Anforderungsniveau zu begrenzen, dass sie mit den an sich schon sehr ressourcenaufwändigen und rigiden Arbeitsabläufen des empirischen Hypothesentestens vereinbar bleibt. Da die psychologische Fachwelt auf die Replikationskrise in erster Linie damit reagierte, die Standards für saubere Methodik zu erhöhen \autocite{van_dongen_productive_2025}, ist es seitdem vielleicht sogar noch schwerer geworden, saubere Theorienentwicklung und saubere empirische Methodik unter einen Hut zu bringen. Vor diesem Hintergrund muss zunächst positiv hervorgehoben werden, dass \textcite{brotzeller_exploring_2025} überhaupt ein theoretisches Argument präsentieren und in Studie 4 überprüfen. Insbesondere, da sie offensichtlich auf keine der von \textcite{sutton_what_1995} identifizierten Kompromissstrategien zurückgegriffen haben.

\Textcite{brotzeller_exploring_2025} erwarten in Studie 4 einen Interaktionseffekt zwischen den wahrgenommenen Verbesserungsmöglichkeiten und der Richtung der Diskrepanz. Laut \textcite{sutton_what_1995} würden viele Autoren vor dem Hintergrund dieser Hypothese Unterschiede in den wahrgenommenen Verbesserungsmöglichkeiten als ausreichende Erklärung für die inkonsistenten Befunde bezüglich der Richtung der Asymmetrie in den Selbstkonzeptänderungen missverstehen. \Citeauthor{brotzeller_exploring_2025} bemühen sich jedoch richtigerweise um eine Erklärung, warum diese Zusammenhänge zu erwarten sind.

\Textcite{brotzeller_exploring_2025} verwenden einen bestehenden Erklärungsansatz aus der Literatur wieder. Anstatt nur auf ihn zu verweisen, führen sie ihn jedoch aus und leiten begründet eine Hypothese aus ihm ab. \Textcite{sutton_what_1995} beobachteten häufig, dass Autoren eine Theorie entwickeln, indem sie empirische Befunde ohne weitere inhaltliche Begründung zueinander in Beziehung setzen. Demgegenüber zitieren \textcite{brotzeller_exploring_2025} empirische Befunde stets zur Fundierung einer inhaltlichen Begründung für den Zusammenhang zwischen Variablen (Beispiel: Die Wahrnehmung geringer Verbesserungsmöglichkeiten provoziert Self"=Enhancement, weil das Annehmen von negativem Feedback unter diesen Umständen zu permanenten negativen Selbstwahrnehmungen führen würde).

\Textcite{sutton_what_1995} zufolge lassen sich gute Theorien nicht nur verbal, sondern auch in Form eines Diagramms darstellen. Das setzt natürlich voraus, dass die theoretischen Annahmen präzise genug formuliert sind, um sie in einem Diagramm darstellen zu können. \Cref{fig:abb-4} belegt, dass die Theorie der Autoren dieses Kriterium erfüllt. 

Ausgangspunkt für die Untersuchungen von \textcite{brotzeller_exploring_2025} ist, dass noch kein Faktor identifiziert wurde, der bestimmt, wann ein Negativbias und wann ein Positivbias auftritt. Ein mögliches Vorgehen wäre es gewesen, den Einfluss möglichst vieler Faktoren auf die Richtung der Asymmetrie zu messen, in der Hoffnung, dass einer dieser Faktoren einen signifikanten Einfluss ausüben würde. \Citeauthor{brotzeller_exploring_2025} gehen allerdings anders vor: Sie stellen erst theoretische Überlegungen darüber an, welche Prozesse einen Positivbias und einen Negativbias erklären würden, und leiten daraus einen in Frage kommenden Faktor ab. \textcite{sutton_what_1995} zufolge gibt es durchaus Stimmen in der Literatur, die sich für das erste Vorgehen aussprechen. Diesen Stimmen zufolge müsse man als Basis für die Entwicklung erfolgreicher Theorien zunächst ein breites empirisches Wissen über möglichst viele Einflussfaktoren und ihr Zusammenwirken aufbauen. Die theoriegeleitete Suche der Autoren nach einem Einflussfaktor ist angesichts der folgenden beiden Vorteile aber mindestens ebenso gerechtfertigt: Sie ist ressourcensparender und erprobt die Erklärungskraft bestehender Erklärungsprinzipien in einem neuen Anwendungsbereich.
