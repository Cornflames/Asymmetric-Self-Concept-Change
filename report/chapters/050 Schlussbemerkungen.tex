\section{Schlussfolgerung}
Die vorliegende Arbeit zeigte die Vorteile der Formalisierung von psychologischen Theorien am Beispiel von \textcite{brotzeller_exploring_2025}. Die Formalisierung der von den Autoren beschriebenen theoretischen Erklärung bildete die Grundlage, um ihre Erklärungskraft noch vor jeder empirischen Überprüfung ihrer Vorhersagekraft zu beurteilen. Gleichzeitig ermöglichte die Formalisierung eine genau Betrachtung, welche Vorhersagen aus der Theorie folgen.

Da die Formalisierung einen zusätzlichen Arbeitsschritt zwischen die verbale Formulierung einer Theorie und ihrer empirischen Überprüfung durch Vorhersagen einfügt, erhöht sie für Forscher den Arbeitsaufwand. Dieser Nachteil der Formalisierung wird aber durch ihre Vorteile überwogen.

Die Formalisierung ermöglicht es, die Erklärungskraft von Theorien zu beurteilen. Diese zentrale Dimension der Güte von Theorien wird durch die bloße Überprüfung ihrer Vorhersagekraft nicht abgedeckt. Wenn Theorien unser Verständnis der Realität erhöhen sollen, dann müssen sie auch eine kohärente und verständliche Beschreibung der Realität darstellen. Außerdem ist das Szenario denkbar, dass eine Theorie richtige Vorhersagen trifft, ohne eine sinnvolle und kohärente Beschreibung der Realität zu sein.

Darüber hinaus ermöglicht die Formalisierung einer Theorie, exaktere Vorhersagen aus ihr abzuleiten, und erhöht so die Aussagekraft ihrer empirischen Überprüfung. Das ist kein Argument gegen verbale Theorien, im Gegenteil: Verbale Beschreibungen sind ungleich verständlicher und intuitiv zugänglicher als formale Modelle. Aber die Evaluation der Präzision und Kohärenz der theoretischen Annahmen im Rahmen ihrer Formalisierung verhilft dazu, verständlichere verbale Beschreibungen der Theorie zu verfassen. Da die Formalisierung von Theorien also die Qualität der bisherigen Schritte des Forschungsprozesses – d.h. die verbale Beschreibung einer Theorie und die empirische Überprüfung von Vorhersagen – erhöht, fügt sie sich nahtlos in die gewohnten Arbeitsabläufe der psychologischen Forschung ein.

\label{sec:end}