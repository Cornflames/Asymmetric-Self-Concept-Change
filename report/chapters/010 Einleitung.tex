\section{Einleitung}
Die Replikationskrise in der Psychologie gab Anlass zu vielen Gegenmaßnahmen, mit denen die psychologische Forschung verbessert werden sollte. Auch wenn diese Maßnahmen in erster Linie die Qualität der empirischen Methodik betrafen, wurden in jüngster Zeit Stimmen lauter, welche die geringe Qualität psychologischer Theorien mit der Replikationskrise in Verbindung bringen und die Verbesserung der Theoriearbeit fordern \autocites{borsboom_theory_2021,van_dongen_productive_2025}. Die psychologische Forschung konzentriert sich vor allem auf die empirische Überprüfung von Vorhersagen und vernachlässigt die systematische Entwicklung guter Theorien, obwohl gute Theorien die Voraussetzung dafür sind, präzise Vorhersagen zu treffen \autocite{borsboom_theory_2021}. Soll die Vorhersage eines Phänomens der Verifizierung einer Theorie dienen, dann ist das Eintreten der Vorhersage nur dann eine valide Aussage über die Gültigkeit der Theorie, wenn die Theorie tatsächlich diese Vorhersage trifft. Umgekehrt gilt: Macht eine Theorie eine bestimmte Vorhersage und tritt diese in der Realität nicht ein, falsifiziert die fehlende Beobachtung der Vorhersage die Theorie nur, wenn die Theorie tatsächlich diese Vorhersage trifft. Da in der Psychologie Theorien meistens nur verbal vorliegen und verbale Formulierungen oft zu unpräzise oder uneindeutig sind, um exakte Vorhersagen aus der Theorie abzuleiten, ist die empirische Überprüfung von Theorien in vielen Fällen nicht aussagekräftig \autocite{van_dongen_productive_2025}.

In diesem Kontext sind die Untersuchungen von \textcite{brotzeller_exploring_2025} zu sehen, die der Erforschung von Selbstkonzeptänderungen nach diskrepantem Feedback dienen. Genauer gesagt untersuchen \Citeauthor{brotzeller_exploring_2025}, welchen Einfluss Eigenschaften von diskrepantem Feedback auf nachfolgende Änderungen des Selbstkonzepts nehmen. Da die bisherige Forschung inkonsistente Befunde bezüglich des Einflusses einer dieser Eigenschaften produziert hat, skizzieren die Autoren eine theoretische Erklärung, welche die Systematik hinter dieser Inkonsistenz aufdecken soll, und überprüfen sie anhand einer Vorhersage, die aus ihr folgt.

Da diese Erklärung, fortan Theorie genannt, nur verbal vorliegt, dient die vorliegende Arbeit der Überprüfung, ob die Autoren ihre theoretischen Annahmen präzise und kohärent genug formuliert haben, um eine exakte Vorhersage treffen zu können. Zu diesem Zweck formalisiert die Arbeit im Teil \enquote{Untersuchung der theoretischen Fundierung} vor dem Hintergrund des Productive"=Explanation"=Framework \autocite{van_dongen_productive_2025} die theoretischen Annahmen der Autoren als Grundlage für ein mögliches formales Modell. Dabei stellt sich unter anderem heraus, dass der Erklärungswert der Theorie der Autoren, selbst wenn sich ihre Vorhersagen bestätigen sollten, als eher gering einzuschätzen ist. Der anschließende Teil \enquote{Reproduktion von Studie 4} dient der Reproduktion der Ergebnisse, mit deren Hilfe \Citeauthor{brotzeller_exploring_2025} die Gültigkeit ihrer theoretischen Erklärung überprüfen wollen. Zwar lassen sich die Ergebnisse reproduzieren, was für die Schlussfolgerung der Autoren spricht, dass die Theorie nicht gültig ist. Allerdings können Aspekte der Theorie diskutiert werden, aufgrund deren sich Zweifel an dieser Schlussfolgerung anmelden lassen. Mit dieser Diskussion beginnt der abschließende Teil der vorliegenden Arbeit. In seinem weiteren Verlauf wird erörtert, wie sich die Erklärungskraft der Theorie außerhalb des Productive"=Explanation"=Framework überprüfen und bewerten lässt.

\label{sec:intro}

