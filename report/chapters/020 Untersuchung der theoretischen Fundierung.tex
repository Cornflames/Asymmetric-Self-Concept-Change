\section{Untersuchung der theoretischen Fundierung}
\label{sec:search}

\Citeauthor{brotzeller_exploring_2025} führen vier Studien durch, in denen sie Änderungen des Selbstkonzepts untersuchen, die Personen als Reaktion auf Feedback vornehmen, das von ihrer Selbstwahrnehmung abweicht \autocite{brotzeller_exploring_2025}. Sie konzentrieren sich darauf, inwiefern die Eigenschaften des diskrepanten Feedbacks nachfolgende Selbstkonzeptänderungen vorhersagen können. Da die bisherige Forschung inkonsistente Befunde bezüglich einer dieser Eigenschaften hervorgebracht hat, skizzieren die Autoren eine mögliche theoretische Erklärung für die Entstehung dieser Inkonsistenz. Diese Erklärung wird im vorliegenden Abschnitt einer formalen Untersuchung unterzogen. 

Die Autoren \citeauthor{brotzeller_exploring_2025} machen kenntlich, dass sie lediglich einen Erklärungsansatz ausführen, der in der bisherigen Literatur bereits kursiert \autocite{brotzeller_exploring_2025}. Die Untersuchung bezieht sich also lediglich auf die Version der Theorie, die die Autoren formulieren.
Da die Theorie noch nicht systematisch falsifiziert wurde, unternehmen diese in Studie 4 einen ersten Falsifizierungsversuch und kommen zu dem Schluss, dass die Theorie das Phänomen nicht erklären kann. Man kann also davon sprechen, dass die Autoren alle Phasen der Entwicklung und Überprüfung der Theorie bereits durchlaufen haben. 

\Citeauthor{brotzeller_exploring_2025} testen die Erklärungskraft der Theorie auf klassischem Wege, nämlich indem sie eine ihrer zentralen Vorhersagen widerlegen \autocite{brotzeller_exploring_2025}. In diesem Abschnitt soll ihre Erklärungskraft vor dem Hintergrund eines anderen Ansatzes zur Beurteilung der Güte von Theorien neu bewertet werden. Als Vorbereitung darauf müssen die Phasen der Theorienentwicklung, die \citeauthor{brotzeller_exploring_2025} bereits durchlaufen haben und deren Ergebnisse sie in ihrem Artikel berichten, noch einmal kritisch nachvollzogen werden.

\Citeauthor{van_dongen_productive_2025} definieren in ihrem Productive Explanation Framework gute Erklärungen als produktive Erklärungen \autocite{van_dongen_productive_2025}. Ursprung dieser Sichtweise ist das folgende Gedankenexperiment: Wenn die Welt so aussähe, wie die Theorie sie beschreibt, dann sollte das Phänomen als stabile Eigenschaft der beobachtbaren Welt ganz automatisch folgen. Dieses Gedankenexperiment kann in Form von Simulationen auch tatsächlich durchgeführt werden. Dafür müssen die verbal vorliegenden qualitativen Beschreibungen von Theorie und Phänomen aber zuerst in ihre exakteren quantitativen Darstellungen überführt werden. Die Theorie kann das Phänomen dann erklären, wenn ein formales Modell, das die wichtigsten Zusammenhänge der Theorie mathematisch abbildet, dazu in der Lage ist, in einer Datensimulation das statistische Muster zu produzieren, das das Phänomen repräsentiert.

Die empirische Falsifizierung von Theorien anhand von Vorhersagen wird erschwert, wenn die theoretischen Annahmen so unpräzise formuliert sind, dass sich keine klaren Vorhersagen ableiten lassen \autocite{van_dongen_productive_2025}. Außerdem kann es bei unpräzisen Formulierungen einer Theorie ebenso vorkommen, dass eine Vorhersage nur deshalb nicht zutrifft, weil sie in Wahrheit gar nicht aus der Theorie folgt. Weil Datensimulationen noch vor jeder empirischen Datenerhebung durchgeführt werden können, kann mit ihnen im Vorhinein überprüft werden, ob die theoretischen Annahmen überhaupt präzise genug formuliert sind, um wenigstens das Phänomen, das sie ursprünglich erklären wollen, in den Daten vorhersagen zu können.
Diesen Schritt überspringen \citeauthor{brotzeller_exploring_2025} und leiten direkt eine Vorhersage aus der Theorie ab \autocite{brotzeller_exploring_2025}. Er soll in diesem Abschnitt nachgeholt werden; dabei werden im besten Fall Unschärfen und Lücken der theoretischen Annahmen sichtbar. 

Die Annahmen der Theorie werden in \cref{sec:search-theory} analysiert. Auf Grundlage dieser Analyse wird in \td der Versuch unternommen, ein formales Modell der Theorie abzuleiten. Um zu überprüfen, ob ein formales Modell das Phänomen produzieren kann, muss dieses zuerst als statistisches Muster repräsentiert werden, was wiederum nur möglich ist, wenn das Phänomen klar definiert ist. Die Präzision der Definition des Phänomens wird in \cref{sec:search-phaenomenon} analysiert. Um dieser Analyse folgen zu können, benötigt man aber zunächst ein grundlegendes Verständnis für das verwendete Untersuchungsparadigma und die darin vorkommenden Variablen.

Um die Präzision und Kohärenz der verbalen Beschreibungen der Autoren zu bewerten, kommt das visuelle Analysetool für Argumentationsstrukturen (VAST) zum Einsatz \autocite{leising_visual_2023}. Mit diesem Tool werden Diagramme erstellt (sog. VAST"=Displays), die verbale Sinnzusammenhänge als Beziehungen zwischen den Konzepten darstellen, die aus den verbalen Beschreibungen herausgelesen werden können. In \cref{tab:annahmetabelle} in \cref{app:table} findet sich eine Auflistung aller Textstellen aus \autocite{brotzeller_exploring_2025}, die für die kommenden Abschnitte relevant sind. Aus jeder Textstelle wurden die wichtigsten Annahmen extrahiert und nummeriert. Zur Vorbereitung auf die Erstellung der VAST"=Displays wurden jeder Annahme die Konzepte zugeordnet, über die sie eine Aussage trifft. Außerdem wurde die Art der Beziehung zwischen diesen Konzepten bewertet, die eine Annahme zum Ausdruck bringt. In den nachfolgenden VAST"=Displays sind die Beziehungspfeile zwischen den dargestellten Konzepten mit den Identifikationsnummern derjenigen Annahmen versehen, auf die sie sich beziehen.

\subsection{Untersuchungsparadigma}
\label{sec:search-paradigma}

Um den Einfluss der Eigenschaften von diskrepantem Feedback auf nachfolgende Selbstkonzeptänderungen zu untersuchen, verwenden \citeauthor{brotzeller_exploring_2025} in den Studien 2 bis 4 ein Untersuchungsparadigma mit den denselben drei Schritten. In Studie 1 durchlaufen die Versuchspersonen diese Schritte nicht während des Experiments, sondern werden gebeten, von einer Episode aus ihrer Vergangenheit zu berichten, in der sie die gleichen Schritte durchlebt haben \autocite{brotzeller_exploring_2025}.

Zuerst werden die Versuchsteilnehmer um eine Selbsteinschätzung bezüglich eines Aspekts ihres Selbstkonzepts gebeten. Genauer gesagt werden sie in den Studien 2 bis 4 darüber befragt, wie sie ihre Fähigkeiten in einem bestimmten Bereich einschätzen. Je nachdem, wie hoch sie ihre Fähigkeiten in diesem Bereich einschätzen, fällt ihre Selbstwahrnehmung mehr oder weniger positiv aus \autocite{brotzeller_exploring_2025}.

Anschließend bekommen die Teilnehmer von außen eine alternative Sicht auf ihre Fähigkeiten gespiegelt, die entweder positiver oder negativer als ihre erste Selbsteinschätzung ist. Diese diskrepante Fremdeinschätzung wird ihnen in Form von Feedback mitgeteilt, das entweder einer realistischen Messung ihrer Fähigkeiten aus einem Test entspringt (Studien 3 und 4) oder präpariert wurde (Studie 2) \autocite{brotzeller_exploring_2025}.

Schenken die Versuchspersonen dem Feedback Glauben, dann zeigt es an, dass sie sich fälschlicherweise über- oder unterschätzt haben, und sollte sie dazu veranlassen, ihre Selbstwahrnehmung dahingehend zu korrigieren. Sie sollten also eine positivere oder negativere Selbstwahrnehmung übernehmen, wenn die Fremdeinschätzung entsprechend positiver oder negativer ausfiel als ihre ursprüngliche Selbsteinschätzung. Man kann dann davon sprechen, dass die Versuchsteilnehmer ihr Selbstkonzept in Richtung des diskrepanten Feedbacks ändern. Um diese Selbstkonzeptänderungen messbar zu machen, werden sie in einem dritten Schritt erneut nach einer (nun aktualisierten) Selbsteinschätzung gefragt \autocite{brotzeller_exploring_2025}.

Je nachdem, wie sehr Personen ihre Selbstwahrnehmung an das Feedback anpassen, fallen die vorgenommenen Änderungen des Selbstkonzepts unterschiedlich groß aus. Ziel der Untersuchungen von \citeauthor{brotzeller_exploring_2025} ist es nun, die Größe der Selbstkonzeptänderungen aus der Größe und der Richtung der Diskrepanz zwischen Feedback und ursprünglicher Selbstwahrnehmung vorherzusagen. Wie sie diese Variablen messen bzw. berechnen, ist in \cref{fig:variables} abgebildet. Diese formale Darstellung mit VAST fällt leicht, weil die Autoren sich um klare Definitionen bemühen.
\begin{figure}[p]
    \centering
    \includegraphics[width=0.9\textwidth]{report/images/Abbildung1_Definitionen_von_Variablen.png}
    \caption{Definition von Variablen.}
    \label{fig:variables}
\end{figure}

Im weiteren Verlauf der Arbeit werden die Formulierungen \enquote{Diskrepantes Feedback annehmen}, \enquote{Feedback in das Selbstkonzept integrieren} und \enquote{Selbstkonzeptänderungen nach diskrepantem Feedback} synonym verwendet. Wird überdies von \enquote{positivem} oder \enquote{negativem} Feedback gesprochen, dann wird die Valenz immer relativ zur ursprünglichen Selbsteinschätzung bemessen. Die beiden Begriffe stehen also stellvertretend für Feedback, das positiver oder negativer als die eigene Selbstwahrnehmung ausfiel.

\subsection{Definition des Phänomens}
\label{sec:search-phaenomenon}

Die Phänomene, die innerhalb des verwendeten Untersuchungsparadigmas auftreten können, lassen sich von unterschiedlichen Graden der Verallgemeinerbarkeit aus betrachten. Das allgemeinste zu erwartende Phänomen wäre, dass Versuchspersonen, die diskrepantes Feedback erhalten, ihr Selbstkonzept in Richtung dieses Feedbacks ändern. Dieses Phänomen wird in \cref{fig:phaenomenons} zuoberst formal dargestellt. Je weiter man sich in diesem VAST"=Display nach unten bewegt, desto spezifischer werden die abgebildeten Phänomene.
\begin{figure}[p]
    \centering
    \includegraphics[height=0.9\textheight]{report/images/Abbildung2_Übersicht_der_Phänomene.png}
    \caption{Übersicht der Phänomene.}
    \label{fig:phaenomenons}
\end{figure}

Die Selbstkonzeptänderungen lassen sich nun getrennt als Folge einer der beiden Eigenschaften des diskrepanten Feedbacks betrachten. Im mittleren Kasten in \cref{fig:phaenomenons} wird deshalb die Diskrepanz zwischen Feedback und Selbstwahrnehmung in ihre Komponenten zerlegt. Die bisherige Forschung konnte als stabiles Phänomen identifizieren, dass größere Diskrepanzen auch zu größeren Anpassungen des Selbstkonzepts führen. Außerdem konnte sie zeigen, dass negatives und positives Feedback i.d.R. zu unterschiedlich großen, d.h. zu asymmetrischen Selbstkonzeptänderungen führen.

Die Darstellung im mittleren Kasten in \cref{fig:phaenomenons} entspricht einem Regressionsmodell, das den Betrag, d.h. die absolute Größe der Selbstkonzeptänderungen, aus der Richtung und der Größe der Diskrepanz sowie der Interaktion dieser beiden Prädiktoren vorhersagt. Daran wird ersichtlich, wie sich die beiden gerade beschriebenen Phänomene als statistische Muster repräsentieren lassen: Als Haupteffekte der Größe und Richtung der Diskrepanz, ausgedrückt als signifikante Steigungsparameter der Prädiktoren. \Citeauthor{brotzeller_exploring_2025} nehmen den Interaktionseffekt beider Prädiktoren mit in ihr Modell auf, weil dieser in der bisherigen Forschung vernachlässigt wurde \autocite{brotzeller_exploring_2025}. Ihre Befunde bezüglich dieses Effekt bleiben allerdings inkonsistent.

Auch wenn bisherige Studien das Phänomen asymmetrischer Selbstkonzeptänderungen nachwiesen, produzierten sie doch inkonsistente Ergebnisse darüber, ob Versuchspersonen negatives oder positives Feedback vermehrt in ihr Selbstkonzept integrieren. Die meisten Studien identifizierten zwar einen Positivbias in den Selbstkonzeptänderungen, aber es finden sich ebenfalls Belege für einen Negativbias. Ausgedrückt als statistisches Muster bedeutet das: Das Vorzeichen des Steigungsparameters der Richtung der Diskrepanz war in den meisten Fällen positiv, in manchen Fällen hingegen negativ.

\Citeauthor{brotzeller_exploring_2025} fassen die Inkonsistenz der Befunde bezüglich der Richtung des Effekts der Größe der Diskrepanz nun wiederum als eigenes Phänomen auf \autocite{brotzeller_exploring_2025}. Die von ihnen skizzierte Theorie soll dieses Phänomen erklären. Die qualitative Beschreibung des Phänomens als \enquote{inkonsistente Befunde} entspricht allerdings nicht dem Phänomen, das die Autoren eigentlich betrachten und erklären wollen, und muss deshalb präzisiert werden. Würde man bei dieser Beschreibung bleiben, ließe sich das Phänomen nicht als statistisches Muster abbilden, denn es wäre lediglich die qualitative Feststellung möglich, dass die Befunde inkonsistent sind.

\Citeauthor{brotzeller_exploring_2025} wollen nicht die Inkonsistenz der Befunde an sich erklären, sondern vielmehr die Systematik hinter dieser Inkonsistenz aufzeigen \autocite{brotzeller_exploring_2025}. Sie gehen davon aus, dass Versuchspersonen in Abhängigkeit von den Parametern des Untersuchungskontexts, die sie wahrnehmen, entscheiden, ob sie eher positives oder eher negatives Feedback vermehrt in ihr Selbstkonzept integrieren. Es gibt also offensichtlich stabile, beobachtbare Zusammenhänge zwischen bestimmten Wahrnehmungen und der Richtung der asymmetrischen Änderungen des Selbstkonzepts.
Diese Zusammenhänge lassen sich als Phänomene definieren und als statistische Effekte repräsentieren. Die Inkonsistenz der Befunde ist dann das beobachtbare Ergebnis davon, dass sich die Versuchspersonen mancher Studien Wahrnehmungen bilden, die mit einem Negativbias assoziiert sind, die Teilnehmer anderer Studien hingegen solche, die mit einem Positivbias zusammenhängen.

\Citeauthor{brotzeller_exploring_2025} bringen Unterschiede darin, wie viele Möglichkeiten Personen innerhalb einer Studie wahrnehmen, sich auf der in Frage stehenden Dimension ihres Selbstkonzepts verbessern zu können, mit der Inkonsistenz der Befunde in Zusammenhang \autocite{brotzeller_exploring_2025}. Sie konzentrieren sich also auf einen bestimmten Aspekt der Wahrnehmung, der den Effekt der Richtung der Diskrepanz auf die Selbstkonzeptänderungen moderiert. Bestimmte Ausprägungen dieser Moderatorvariable sind mit einem negativen Vorzeichen des Effekts (d.h. mit einem Negativbias) assoziiert, gewisse andere Ausprägungen hingegen mit einem positiven Vorzeichen (d.h. mit einem Positivbias). Die Theorie der Autoren hat zum Ziel, vermittelnde Prozesse zu spezifizieren, die diese Zusammenhänge erklären. Das Phänomen, das die Theorie erklären will, lässt sich nun präziser definieren; in \cref{fig:placeholder-3} wird diese neue Definition formal dargestellt.
\begin{figure}[p]
    \centering
    \includegraphics[height=0.9\textheight]{report/images/Abbildung2_Übersicht_der_Phänomene.png}
    \caption{\textcolor{red}{Abbildung 3 fehlt!}}
    \label{fig:placeholder-3}
\end{figure}

Das Phänomen lässt sich in zwei Schritten als statistisches Muster repräsentieren. Zunächst müsste ein grundsätzlicher Moderationseffekt über einen Interaktionseffekt zwischen den wahrgenommenen Verbesserungsmöglichkeiten und der Richtung der Diskrepanz bei der Vorhersage von Selbstkonzeptänderungen beobachtbar sein (oberer Kasten in \cref{fig:placeholder-3}). Weiterhin müssten stabile Zusammenhänge zwischen bestimmten Ausprägungen des Moderators und einem Negativ- bzw. Positivbias beobachtbar sein (unterer Kasten in \cref{fig:placeholder-3}). \Citeauthor{brotzeller_exploring_2025} versuchen in Studie 4, die Existenz des Phänomens mit diesen beiden Schritten nachzuweisen \autocite{brotzeller_exploring_2025}.

Die Theorie muss psychologische Prozesse spezifizieren, die zwischen Moderator und Asymmetrien vermitteln. Der Moderator zählt nicht zu diesen Prozessen, sondern ist Teil des Phänomens. Er selbst ist also keine Erklärung für den Richtungswechsel der asymmetrischen Selbstkonzeptänderungen, ebenso wenig wie der Stand des Mondes und der Sonne den Ebbe-Flut-Zyklus per se erklären -- vielmehr sind es die Gravitationskräfte, die bei einem bestimmten Zusammenstand beider Himmelskörper eine der beiden Gezeiten herbeiführen.

\subsection{Analyse der theoretischen Annahmen}
\label{sec:search-theory}

% \subsection{Ein subjektiver Schwellenwert für die Wahrnehmung von Verbesserungsmöglichkeiten}
% \label{sec:search-thresh}

% \subsection{Die Evaluationsmotive als logische Folgen niedriger und hoher Verbesserungsmöglichkeiten}
% \label{sec:search-eval}

% \subsection{Der Positivbias und der Negativbias als logische Folgen der Evaluationsmotive}
% \label{sec:search-bias}
