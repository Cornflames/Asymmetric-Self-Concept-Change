\section{Untersuchung der theoretischen Fundierung}
\label{sec:search}

\Textcite{brotzeller_exploring_2025} führen vier Studien durch, in denen sie Änderungen des Selbstkonzepts untersuchen, die Personen als Reaktion auf Feedback vornehmen, das von ihrer Selbstwahrnehmung abweicht. Sie konzentrieren sich darauf, inwiefern die Eigenschaften des diskrepanten Feedbacks nachfolgende Selbstkonzeptänderungen vorhersagen können. Da die bisherige Forschung inkonsistente Befunde bezüglich einer dieser Eigenschaften hervorgebracht hat, skizzieren die Autoren eine mögliche theoretische Erklärung für die Entstehung dieser Inkonsistenz.

\Textcite{brotzeller_exploring_2025} machen kenntlich, dass sie lediglich einen Erklärungsansatz ausführen, der in der bisherigen Literatur bereits kursiert. Die folgende formale Untersuchung bezieht sich also lediglich auf die Version der Theorie, die die Autoren formulieren.
Da die Theorie noch nicht systematisch falsifiziert wurde, unternehmen \Citeauthor{brotzeller_exploring_2025} in Studie 4 einen ersten Falsifizierungsversuch und kommen zu dem Schluss, dass die Theorie das Phänomen nicht erklären kann. Man kann also davon sprechen, dass die Autoren alle Phasen der Entwicklung und Überprüfung der Theorie bereits durchlaufen haben. 

\Textcite{brotzeller_exploring_2025} testen die Erklärungskraft der Theorie auf klassischem Wege, indem sie eine ihrer zentralen Vorhersagen widerlegen. In diesem Abschnitt soll ihre Erklärungskraft vor dem Hintergrund eines anderen Ansatzes zur Beurteilung der Güte von Theorien neu bewertet werden. Als Vorbereitung darauf müssen die Phasen der Theorieentwicklung, die \citeauthor{brotzeller_exploring_2025} bereits durchlaufen haben und deren Ergebnis sie in ihrem Artikel berichten, noch einmal kritisch nachvollzogen werden.

\Textcite{van_dongen_productive_2025} definieren in ihrem Productive-Explanation-Framework gute Erklärungen als produktive Erklärungen. Ausgangspunkt dieser Definition ist das folgende Gedankenexperiment: Wenn die Welt so aussähe, wie die Theorie sie beschreibt, dann sollte das Phänomen als stabile Eigenschaft der beobachtbaren Welt von alleine folgen. Dieses Gedankenexperiment kann in Form von Simulationen durchgespielt werden. Dafür müssen die verbal vorliegenden qualitativen Beschreibungen von Theorie und Phänomen aber zuerst in exaktere quantitative Darstellungen überführt werden. Die Theorie kann das Phänomen dann erklären, wenn ein formales Modell, das die wichtigsten Zusammenhänge der Theorie mathematisch abbildet, dazu in der Lage ist, in einer Datensimulation das statistische Muster zu produzieren, welches das Phänomen repräsentiert.

Die empirische Verifizierung von Theorien anhand von bestätigten Vorhersagen wird erschwert, wenn die theoretischen Annahmen so unpräzise formuliert sind, dass sich keine klaren Vorhersagen ableiten lassen \autocite{van_dongen_productive_2025}. Außerdem kann es bei unpräzisen Formulierungen einer Theorie ebenso vorkommen, dass eine Vorhersage nur deshalb nicht zutrifft, weil sie in Wahrheit gar nicht aus der Theorie folgt. Mit Datensimulationen kann vor der empirischen Überprüfung einer Vorhersage getestet werden, ob sich diese Vorhersage überhaupt aus der Theorie ableiten lässt. Kann das formale Modell einer Theorie das statistische Muster des Phänomens produzieren, folgt die Vorhersage aus der Theorie.

Bis jetzt wurde das Szenario beleuchtet, dass aus einer bereits vorliegenden Theorie eine Vorhersage abgeleitet und überprüft wird. Datensimulationen nach der Idee des Productive-Explanation-Framework eignen sich aber auch dazu, zu überprüfen, wie hoch die Erklärungskraft einer Theorie ist, die zur Erklärung eines bereits beobachtbaren Phänomens entwickelt wurde. Wenn die Theorie das Phänomen erklärt, also eine treffende Beschreibung der Realität ist, dann müsste ein formales Modell der Theorie das statistische Muster des Phänomens produzieren können; denn das Phänomen müsste sich als Vorhersage aus der Theorie ableiten lassen.

Im Fall von \textcite{brotzeller_exploring_2025} liegt eine Mischung beider Szenarien vor. Das Phänomen, das die Autoren erklären wollen, ist die Inkonsistenz der Befunde bezüglich des Einflusses einer Eigenschaft von diskrepantem Feedback auf nachfolgende Selbstkonzeptänderungen. Da dieser Inkonsistenz aber offensichtlich eine Systematik zu Grunde liegt, muss die Definition des Phänomens so weit präzisiert werden, dass sie einen Faktor enthält, dessen Varianz für die Varianz der Befunde verantwortlich ist. Die Theorie der Autoren dient also der Erklärung dieses neu definierten Phänomens. Da die Autoren aber erst aus ihrer theoretischen Erklärung für die Inkonsistenz der Befunde ableiten, welcher Faktor wahrscheinlich mit den Befunden kovariiert, ist das Phänomen zunächst noch eine Vorhersage der Theorie.

In diesem Kapitel wird zunächst herausgearbeitet, welche Vorhersage die Autoren genau aus der entwickelten Theorie ableiten. Das setzt allerdings ein grundlegendes Verständnis für das verwendete Untersuchungsparadigma und die wichtigsten Variablen voraus, das im ersten Abschnitt vermittelt wird. Nachdem die genaue Vorhersage der Autoren identifiziert und in eine formale Definition des zu erwartenden Phänomens überführt wurde, werden die Annahmen der theoretischen Erklärung für dieses Phänomen formalisiert und auf ihre Kohärenz und Präzision hin geprüft. Das Kapitel endet mit einem Fazit, ob die Theorie eine gute Erklärung für das Phänomen wäre, wenn man davon ausgeht, dass dieses Phänomen tatsächlich beobachtbar ist. Darin kommt gleichzeitig zum Ausdruck, ob sich das Phänomen als Vorhersage aus der Theorie ableiten lässt.

Um die Präzision und Kohärenz der verbalen Beschreibungen der Autoren zu bewerten, kommt das visuelle Analysetool für Argumentationsstrukturen (VAST) zum Einsatz \autocite{leising_visual_2023}. Mit diesem Tool werden Diagramme erstellt (sog. VAST"=Displays), die verbale Sinnzusammenhänge als Beziehungen zwischen den Konzepten darstellen, die aus den verbalen Beschreibungen herausgelesen werden können. In \cref{tab:annahmetabelle} in \cref{app:table} findet sich eine Auflistung aller Textstellen aus \textcite{brotzeller_exploring_2025}, die für die kommenden Abschnitte relevant sind. Aus jeder Textstelle wurden die wichtigsten Annahmen extrahiert und nummeriert. Zur Vorbereitung auf die Erstellung der VAST"=Displays wurden jeder Annahme die Konzepte zugeordnet, über die sie eine Aussage trifft. Außerdem wurde die Art der Beziehung zwischen den Konzepten bewertet, die eine Annahme zum Ausdruck bringt. In den nachfolgenden VAST"=Displays sind die Beziehungspfeile zwischen den dargestellten Konzepten mit den Identifikationsnummern derjenigen Annahmen versehen, auf die sie sich beziehen.

\subsection{Untersuchungsparadigma}
\label{sec:search-paradigma}

Um den Einfluss der Eigenschaften von diskrepantem Feedback auf nachfolgende Selbstkonzeptänderungen zu untersuchen, verwenden \textcite{brotzeller_exploring_2025} in den Studien 2 bis 4 ein Untersuchungsparadigma mit denselben drei Schritten. In Studie 1 durchlaufen die Versuchspersonen diese Schritte nicht während des Experiments, sondern werden gebeten, von einer Episode aus ihrer Vergangenheit zu berichten, in der sie die gleichen Schritte durchlebt haben.

Zuerst sollen die Versuchsteilnehmer eine Selbsteinschätzung bezüglich eines Aspekts ihres Selbstkonzepts abgeben. Genauer gesagt werden sie in den Studien 2 bis 4 darüber befragt, wie sie ihre Fähigkeiten in einem bestimmten Bereich einschätzen. Je nachdem, wie hoch sie ihre Fähigkeiten in diesem Bereich einschätzen, fällt ihre Selbstwahrnehmung mehr oder weniger positiv aus.

Anschließend bekommen die Teilnehmer von außen eine alternative Sicht auf ihre Fähigkeiten gespiegelt, die entweder positiver oder negativer als ihre erste Selbsteinschätzung ist. Diese diskrepante Fremdeinschätzung wird ihnen in Form von Feedback mitgeteilt, das entweder einer realistischen Messung ihrer Fähigkeiten aus einem Test entspringt (Studien 3 und 4) oder präpariert wurde (Studie 2).

Schenken die Versuchspersonen dem Feedback Glauben, dann zeigt es an, dass sie sich fälschlicherweise über- oder unterschätzt haben, und sollte sie dazu veranlassen, ihre Selbstwahrnehmung dementsprechend zu korrigieren. Sie sollten also eine positivere oder negativere Selbstwahrnehmung übernehmen, wenn die Fremdeinschätzung entsprechend positiver oder negativer ausfiel als ihre ursprüngliche Selbsteinschätzung. Man kann dann davon sprechen, dass die Versuchsteilnehmer ihr Selbstkonzept in Richtung des diskrepanten Feedbacks ändern. Um diese Selbstkonzeptänderungen messbar zu machen, werden sie in einem dritten Schritt erneut nach einer (nun aktualisierten) Selbsteinschätzung gefragt.

Je nachdem, wie sehr Personen ihre Selbstwahrnehmung an das Feedback anpassen, fallen die vorgenommenen Änderungen des Selbstkonzepts unterschiedlich groß aus. Ziel der Untersuchungen von \textcite{brotzeller_exploring_2025} ist es nun, die Größe der Selbstkonzeptänderungen aus der Größe und der Richtung der Diskrepanz zwischen Feedback und ursprünglicher Selbstwahrnehmung vorherzusagen. Wie sie diese Variablen messen bzw. berechnen, ist in \cref{fig:variables} abgebildet. Diese formale Darstellung mit VAST fällt leicht, weil die Autoren sich um klare Definitionen bemühen.
\begin{figure}[p]
    \centering
    \includegraphics[width=0.9\textwidth]{report/images/Abbildung1_Definitionen_von_Variablen.png}
    \caption[Definitionen von Variablen und Konzepten.]{Definitionen von Variablen und Konzepten. Legende Rel.: n = Naming, i = Conceptual Implication, m = Metamorphosis (X turns into Y over time), e = Element of (X partly constitutes Y).}
    \label{fig:variables}
\end{figure}

Im weiteren Verlauf der Arbeit werden die Formulierungen \enquote{Diskrepantes Feedback annehmen}, \enquote{Feedback in das Selbstkonzept integrieren} und \enquote{Selbstkonzeptänderungen nach diskrepantem Feedback} synonym verwendet. Wird überdies von \enquote{positivem} oder \enquote{negativem} Feedback gesprochen, dann wird die Valenz immer relativ zur ursprünglichen Selbsteinschätzung bemessen. Die beiden Begriffe stehen also stellvertretend für Feedback, das positiver oder negativer als die eigene Selbstwahrnehmung ausfiel.

\subsection{Definition des Phänomens}
\label{sec:search-phaenomenon}

Die Phänomene, die innerhalb des verwendeten Untersuchungsparadigmas auftreten können, lassen sich von unterschiedlichen Graden der Verallgemeinerbarkeit aus betrachten. Das allgemeinste zu erwartende Phänomen wäre, dass Versuchspersonen, die diskrepantes Feedback erhalten, ihr Selbstkonzept in Richtung dieses Feedbacks ändern. Dieses Phänomen wird in \cref{fig:phaenomenons} zuoberst formal dargestellt. Je weiter man sich in diesem VAST"=Display nach unten bewegt, desto spezifischer werden die abgebildeten Phänomene.
\begin{figure}[p]
    \centering
    \includegraphics[height=0.85\textheight]{report/images/Abbildung2_Übersicht_der_Phänomene.png}
    \caption[Übersicht der Phänomene.]{Übersicht der Phänomene. Legende Rel.: n = Naming, i = Conceptual Implication, c = Causation, p = Prediction, t = Transformation.}
    \label{fig:phaenomenons}
\end{figure}

Die Selbstkonzeptänderungen lassen sich nun getrennt als Folge einer der beiden Eigenschaften des diskrepanten Feedbacks betrachten. Im mittleren Kasten in \cref{fig:phaenomenons} wird deshalb die Diskrepanz zwischen Feedback und Selbstwahrnehmung in ihre Komponenten zerlegt. Die bisherige Forschung konnte als stabiles Phänomen identifizieren, dass größere Diskrepanzen auch zu größeren Anpassungen des Selbstkonzepts führen. Außerdem konnte sie zeigen, dass negatives und positives Feedback i.d.R. zu unterschiedlich großen, d.h. zu asymmetrischen Selbstkonzeptänderungen führen.

Die Darstellung im mittleren Kasten in \cref{fig:phaenomenons} entspricht einem Regressionsmodell, das den Betrag, d.h. die absolute Größe der Selbstkonzeptänderungen, aus der Richtung und der Größe der Diskrepanz sowie der Interaktion dieser beiden Prädiktoren vorhersagt. Daran wird ersichtlich, wie sich die beiden gerade beschriebenen Phänomene als statistische Muster repräsentieren lassen: Als Haupteffekte der Größe und Richtung der Diskrepanz, ausgedrückt als signifikante Steigungsparameter der Prädiktoren. \textcite{brotzeller_exploring_2025} nehmen den Interaktionseffekt beider Prädiktoren mit in ihre Modelle auf, weil dieser in der bisherigen Forschung vernachlässigt wurde. Ihre Befunde bezüglich dieses Effekts bleiben allerdings inkonsistent.

Auch wenn bisherige Studien das Phänomen asymmetrischer Selbstkonzeptänderungen nachwiesen, produzierten sie doch inkonsistente Ergebnisse darüber, ob Versuchspersonen negatives oder positives Feedback vermehrt in ihr Selbstkonzept integrieren \autocite{brotzeller_exploring_2025}. Die meisten Studien identifizierten zwar einen Positivbias in den Selbstkonzeptänderungen, aber es finden sich ebenfalls Belege für einen Negativbias. Ausgedrückt als statistisches Muster bedeutet das: Das Vorzeichen des Steigungsparameters der Richtung der Diskrepanz war in den meisten Fällen positiv, in manchen Fällen hingegen negativ.

\Textcite{brotzeller_exploring_2025} fassen die Inkonsistenz der Befunde bezüglich der Richtung des Effekts der Richtung der Diskrepanz nun wiederum als eigenes Phänomen auf. Die von ihnen skizzierte Theorie soll dieses Phänomen erklären. Die qualitative Beschreibung des Phänomens als \enquote{inkonsistente Befunde} entspricht allerdings nicht dem Phänomen, das die Autoren eigentlich betrachten und erklären wollen, und muss deshalb präzisiert werden. Würde man bei dieser Beschreibung bleiben, ließe sich das Phänomen nicht als statistisches Muster abbilden, denn es wäre lediglich die qualitative Feststellung möglich, dass die Befunde inkonsistent sind.

\Textcite{brotzeller_exploring_2025} wollen nicht die Inkonsistenz der Befunde an sich erklären, sondern vielmehr die Systematik hinter dieser Inkonsistenz aufzeigen. Sie gehen davon aus, dass Versuchspersonen in Abhängigkeit von den Parametern des Untersuchungskontexts, die sie wahrnehmen, entscheiden, ob sie eher positives oder eher negatives Feedback vermehrt in ihr Selbstkonzept integrieren. Es gibt also offensichtlich stabile, beobachtbare Zusammenhänge zwischen bestimmten Wahrnehmungen und der Richtung der asymmetrischen Änderungen des Selbstkonzepts.
Diese Zusammenhänge lassen sich als Phänomene definieren und als statistische Effekte repräsentieren. Die Inkonsistenz der Befunde ist dann das beobachtbare Ergebnis davon, dass sich die Versuchspersonen mancher Studien Wahrnehmungen bilden, die mit einem Negativbias assoziiert sind, die Teilnehmer anderer Studien hingegen solche, die mit einem Positivbias zusammenhängen.

\Citeauthor{brotzeller_exploring_2025} bringen Unterschiede darin, wie viele Möglichkeiten Personen innerhalb einer Studie wahrnehmen, sich auf der in Frage stehenden Dimension ihres Selbstkonzepts verbessern zu können, mit der Inkonsistenz der Befunde in Zusammenhang. Sie konzentrieren sich also auf einen bestimmten Aspekt der Wahrnehmung, der den Effekt der Richtung der Diskrepanz auf die Selbstkonzeptänderungen moderiert. Bestimmte Ausprägungen dieser Moderatorvariable sind mit einem negativen Vorzeichen des Effekts (d.h. mit einem Negativbias) assoziiert, gewisse andere Ausprägungen hingegen mit einem positiven Vorzeichen (d.h. mit einem Positivbias). Die Theorie der Autoren hat zum Ziel, vermittelnde Prozesse zu spezifizieren, die diese Zusammenhänge erklären. Das Phänomen, das die Theorie erklären will, lässt sich nun präziser definieren; in \cref{fig:placeholder-3} wird diese neue Definition formal dargestellt.
\begin{figure}[p]
    \centering
    \includegraphics[width=0.9\linewidth]{report/images/Abbildung3_Definition_des_Phänomens.png}
    \caption[Definition des Phänomens.]{Definition des Phänomens. Legende Rel.: n = Naming, i = Conceptual Implication, p = Prediction.}
    \label{fig:placeholder-3}
\end{figure}

Das Phänomen lässt sich in zwei Schritten als statistisches Muster repräsentieren. Zunächst müsste ein grundsätzlicher Moderationseffekt über einen Interaktionseffekt zwischen den wahrgenommenen Verbesserungsmöglichkeiten und der Richtung der Diskrepanz bei der Vorhersage von Selbstkonzeptänderungen beobachtbar sein (oberer Kasten in \cref{fig:placeholder-3}). Weiterhin müssten sich stabile Zusammenhänge zwischen bestimmten Ausprägungen des Moderators und einem Negativ- bzw. Positivbias beobachten lassen (unterer Kasten in \cref{fig:placeholder-3}). \textcite{brotzeller_exploring_2025} versuchen in Studie 4, die Existenz des Phänomens mit diesen beiden Schritten nachzuweisen.

Die Theorie muss psychologische Prozesse spezifizieren, die zwischen Moderator und Asymmetrien vermitteln. Der Moderator zählt nicht zu diesen Prozessen, sondern ist Teil des Phänomens. Er selbst ist also keine Erklärung für den Richtungswechsel der asymmetrischen Selbstkonzeptänderungen, ebenso wenig wie der Stand des Mondes und der Sonne den Ebbe-Flut-Zyklus per se erklären -- vielmehr sind es die Gravitationskräfte, die bei einem bestimmten Zusammenstand beider Himmelskörper eine der beiden Gezeiten herbeiführen.

\subsection{Analyse der theoretischen Annahmen}
\label{sec:search-theory}

In diesem Abschnitt werden die Annahmen der von \textcite{brotzeller_exploring_2025} skizzierten Theorie analysiert. Grundlage für diese Analyse bildet \cref{tab:annahmetabelle}, in der die theoretischen Annahmen systematisch aus den entsprechenden Textstellen im Originalartikel extrahiert und anschließend nummeriert wurden. Aus diesen Annahmen setzen die Autoren die Theorie im Fließtext Stück für Stück zu einem Ganzen zusammen. Die Reihenfolge ist dabei nicht zufällig gewählt, sondern folgt implizit einem Gedankengang, dem die Autoren wahrscheinlich auch bei der Entwicklung der Theorie gefolgt sind. Sie beginnen damit, bestimmte Grundannahmen zu treffen, die ihrer Erklärung zugrunde liegen. Man könnte diesen Schritt auch als Entwicklung einer allgemeinen \enquote{Idee} der Erklärung bezeichnen. Nimmt man diesen ersten Schritt bei der Entwicklung der Theorie, ergeben sich logische Anschlussfragen, wie die getroffenen Grundannahmen weiter spezifiziert werden müssen, damit eine vollständige und schlüssige Erklärung entsteht. Diese Fragen beantworten die Autoren dann im weiteren Verlauf des Texts.

Als Aufhänger für die Analyse der theoretischen Annahmen dient die Analyse des Gedankengangs, den die Autoren verfolgen. Dafür werden zunächst die Grundannahmen der Theorie herausgearbeitet und anschließend alle Fragen aneinandergereiht, die daraufhin beantwortet werden müssen. Unter jeder Frage werden die Antworten, die \textcite{brotzeller_exploring_2025} geben, mit Verweis auf die Identifikationsnummern der zugehörigen Annahmen in \cref{tab:annahmetabelle} aufgeführt. Bleiben Fragen unbeantwortet, deutet das auf Lücken in der Erklärung hin. Liefern die unter einer Frage zitierten Annahmen keine befriedigende Antwort, ist dieser Teil der Erklärung nicht schlüssig.

\newpage
\subsubsection{Grundannahmen}

Laut \textcite{borsboom_theory_2021} ist eine mögliche Herangehensweise bei der Entwicklung einer Theorie, Erklärungsprinzipien wiederzuverwenden, die in anderen Theorien Anwendung finden. Tatsächlich bedienen sich \textcite{brotzeller_exploring_2025} Prinzipien, die in der Literatur zur Erklärung von Selbstevaluationsprozessen herangezogen werden, wobei es zweifelhaft ist, ob diese einzelnen Theorien zugeordnet werden können.

Die Literatur zur Selbstevaluation nimmt an, dass Personen kein starres Selbstbild besitzen, sondern ihr Selbstkonzept mit Informationen aus ihrer sozialen Umwelt verhandeln. Selbstevaluationsprozesse beschreiben dann, wie Personen selbstbezogene Informationen filtern, ihren Wahrheitsgehalt beurteilen und darauf aufbauend Rückschlüsse über sich ziehen und handeln \autocite{sedikides_self-evaluation_1997}. \Textcite{brotzeller_exploring_2025} betrachten Selbstkonzeptänderungen nun offensichtlich als Ergebnis eines Selbstevaluationsprozesses, der durch diskrepantes und für das Selbstkonzept relevantes Feedback ausgelöst wird.

Diese Grundannahme ist nur unter einer bestimmten Voraussetzung gültig, welche die Autoren auch explizit benennen. So müssen Personen das Feedback, das sie erhalten, als valide Aussage über ihre eigene Person auffassen, damit es Selbstevaluationsprozesse auslöst. \Textcite{brotzeller_exploring_2025} gehen davon aus, dass das verwendete Studiendesign diese Voraussetzung erfüllt, also dass die Versuchsteilnehmer das Feedback als „diagnostisch relevant“ wahrnahmen. Ob diese Voraussetzung tatsächlich erfüllt ist, hängt also von Designentscheidungen ab, von denen eine nach näherer Beachtung verlangt.

Ein Einflussfaktor auf die diagnostische Relevanz ist die Zentralität der angesprochenen Dimension im Selbstkonzept. Spricht das Feedback keinen zentralen, sondern einen eher peripheren (d.h. weniger relevanten) Aspekt des Selbstkonzepts an, sollte es Personen weniger wichtig sein, in diesem Bereich eine positive Selbstwahrnehmung zu entwickeln. Umgekehrt sollte von diesem Feedback keine große Bedrohung für ein positives Selbstkonzept ausgehen. In den Experimenten 2, 3 und 4 bei \textcite{brotzeller_exploring_2025} erhielten Personen Feedback auf Grundlage von Tests, die sehr spezielle Fähigkeitsbereiche mit geringer Alltagsrelevanz abfrugen. Feedback aus solchen \enquote{novel tasks} sollte jedoch keinen zentralen Aspekt des Selbstkonzepts ansprechen \autocite{sedikides_selfimprovement_2009}. Diese Überlegung ist deshalb so wichtig, weil sowohl der Wunsch, positive Selbstwahrnehmungen zu entwickeln, als auch die Bedrohlichkeit von negativem Feedback zentrale Bestandteile der Theorie der Autoren sind.

Die Forschung zur Selbstevaluation geht davon aus, dass Selbstevaluationsprozesse dem grundsätzlichen Ziel dienen, ein möglichst positives Selbstbild aufrechtzuerhalten. Nun kann dieses Ziel auf unterschiedlichen Wegen erreicht werden und dementsprechend lassen sich verschiedene Muster von Evaluationsprozessen unterscheiden. Diese Muster werden sogenannten Evaluationsmotiven zugeordnet, die dem grundsätzlichen Wunsch nach einem positiven Selbstbild entspringen und zum Ausdruck bringen, auf welche Weise Personen dieses Ziel erreichen wollen. So können Personen u.a. motiviert sein, sich zu verbessern (Self"=Improvement"=Motiv), oder sie wollen sich in einem guten Licht darstellen, indem sie sich auf positive Informationen konzentrieren und negative Informationen vermeiden (Self"=Enhancement"=Motiv).

\Textcite{brotzeller_exploring_2025} nehmen an, dass sich der Negativbias und der Positivbias jeweils durch das Wirken eines Evaluationsmotivs erklären lassen. Dass in manchen Experimenten ein Negativitätsbias gefunden wurde, in anderen hingegen ein Positivbias, liegt daran, dass in den Experimenten ein anderes Evaluationsmotiv aktiv war. Die beiden Motive, die die Autoren zur Erklärung heranziehen, sind das Self"=Improvement"=Motiv und das Self"=Enhancement"=Motiv.

Auf Grundlage der bisherigen Überlegungen lässt sich ableiten, welche Anschlussfragen im Weiteren beantwortet werden müssen, damit eine vollständige Erklärung entsteht. Zunächst muss gefragt werden, welches Evaluationsmotiv einen Positivbias und welches Motiv einen Negativbias erklären kann (\labelcref{item:f1} \& \labelcref{item:f2}). Die gegebene Auswahl muss daraufhin inhaltlich begründet werden (\labelcref{item:f3} \& \labelcref{item:f4}). Um erklären zu können, warum in manchen Experimenten das eine, in anderen Experimenten hingegen das andere Evaluationsmotiv aktiv war, muss nach Unterschieden zwischen den Experimenten gesucht werden, die das Auftreten des einen oder anderen Motivs begünstigt haben dürften. Bei dieser Suche werden die Autoren von theoretischen Überlegungen und bisherigen empirischen Befunden geleitet, welche Einflussfaktoren die Aktivierung des einen oder anderen Motivs generell begünstigen sollten (\labelcref{item:f5} \& \labelcref{item:f6}). Der Einfluss der identifizierten Faktoren muss anschließend inhaltlich begründet werden (\labelcref{item:f7} \& \labelcref{item:f8}).

\subsubsection{Ein allgemeines Erklärungsgesetz}

\textcite{brotzeller_exploring_2025} begeben sich auf die Suche nach Faktoren, die die Aktivierung des einen oder anderen Evaluationsmotivs begünstigen. Diese Faktoren sind weniger Weichenstellungen eines bloßen Zufallsprozesses als Parameter des Untersuchungskontexts, die Versuchsteilnehmer berücksichtigen und bei einer überlegten Entscheidung für das eine oder andere Motiv einbeziehen. 

Evaluationsmotive (und die von ihnen ausgelösten Evaluationsprozesse) basieren auf dem Wunsch nach einem möglichst positiven Selbstbild. \Textcite{sedikides_self-evaluation_1997} entwerfen deshalb ein Modell der Selbstevaluation, in dem sich Personen immer für jenes Motiv entscheiden, das Evaluationsprozesse auslöst, die unter den gegebenen Umständen am geeignetsten erscheinen, um ein positives Selbstkonzept aufrechtzuerhalten. Ihrem Modell zufolge ist jedes Motiv in manchen Situationen adaptiver als in anderen, was bedeutet, dass die von ihm ausgelösten Evaluationsprozesse unterschiedlich effektiv darin sind, ein positives Selbstbild aufrechtzuerhalten. Es sollte sich nun immer jenes Motiv durchsetzen, das in der aktuellen Situation am adaptivsten, d.h. am effektivsten ist.

Diesem Modell zufolge würden Personen zunächst die Eigenschaften der aktuellen Situation analysieren und die Effektivität der in Frage kommenden Evaluationsmotive unter diesen Umständen bestimmen. Dann würden sie die Adaptivität der Motive vergleichen und sich für das adaptivste Motiv entscheiden. Auch wenn \textcite{brotzeller_exploring_2025} explizit benennen, dass sowohl Self"=Improvement als auch Self"=Enhancement auf den Wunsch nach einem positiven Selbstbild zurückgehen, bleibt offen, ob sie den Aktivierungsprozess der Motive auch wirklich als Abwägung der Effektivität beider Motive betrachten. Das wäre allerdings wünschenswert, weil diese Entscheidungsregel ein allgemeines Gesetz darstellen würde.

\newpage
Ein allgemeines Gesetz ist ein Erklärungsprinzip, das potenziell noch weitere Phänomene erklären kann, die ursprünglich kein Ziel der theoretischen Erklärung waren. Das Modell von \textcite{sedikides_self-evaluation_1997} ist so allgemein, dass es eine Vielzahl an Evaluationsmustern nach diagnostisch relevantem Feedback erklären kann und darf deshalb als solches Gesetz gelten. Da in der Definition theoretischer Erklärungen nach \textcite{borsboom_theory_2021} jede Theorie mindestens ein solches Gesetz enthalten muss, wird bei der Diskussion des Aktivierungsprozesses der Motive geprüft, ob die Annahmen von \textcite{brotzeller_exploring_2025} mit diesem Modell kompatibel sind.

\subsubsection{Anschlussfragen}

\begin{enumerate}[label=F\arabic*]
    % F1
    \item{
    Welches Evaluationsmotiv bewirkt einen Positivbias in den Änderungen des Selbstkonzepts nach diskrepantem Feedback?
    }\label{item:f1}
    \begin{enumerate}[label=A\arabic*]
        \item{
        Die Evaluationsprozesse, die durch ein Self"=Enhancement"=Motiv ausgelöst werden, sollten zu einem Positivbias in den Änderungen des Selbstkonzepts führen [\hyperlink{id30}{30}]. Self"=Enhancement geht auf den Wunsch zurück, ein positives Selbstbild aufrechtzuerhalten [\hyperlink{id25}{25}], und versucht dieses Ziel durch selektive oder verzerrte Wahrnehmung und Verarbeitung von Informationen zugunsten einer positiven Selbstwahrnehmung zu erreichen [\hyperlink{id22}{22}]. So bevorzugen Personen unter einem Self"=Enhancement"=Motiv z. B. Abwärtsvergleiche, weil sie bei diesen Vergleichen wahrscheinlich gut abschneiden werden.
        }\label{item:a1}
    \end{enumerate}

    % F2
    \item{
    Welches Evaluationsmotiv bewirkt einen Negativbias in den Änderungen des Selbstkonzepts nach diskrepantem Feedback?
    }\label{item:f2}
    \begin{enumerate}[label=A\arabic*,start=2]
        \item{
        Die Evaluationsprozesse, die durch ein Self"=Improvement"=Motiv ausgelöst werden, sollten zu einem Negativbias in den Änderungen des Selbstkonzepts führen [\hyperlink{id33}{33}]. Self"=Improvement geht auf den Wunsch zurück, ein positives Selbstbild aufrechtzuerhalten [\hyperlink{id26}{26}], und versucht dieses Ziel durch aktive Verbesserung zu erreichen. Self"=Improvement"=Prozesse können als Verhaltensweisen und Wahrnehmungen definiert werden, die darauf abzielen, die Diskrepanz zwischen der aktuellen Selbstwahrnehmung (dem Ist"=Zustand) und einer angestrebten positiveren Selbstwahrnehmung (dem Soll"=Zustand) zu reduzieren [\hyperlink{id23}{23}].
        }\label{item:a2}
    \end{enumerate}

    % F3
    \item{
    Warum führt ein Self"=Enhancement"=Motiv zu einem Positivbias?
    }\label{item:f3}
    \begin{enumerate}[label=A\arabic*,start=3]
        \item{
        Sind Personen zu Self-Enhancement motiviert, dann nehmen sie negatives Feedback als Bedrohung für ein positives Selbstkonzept wahr [\hyperlink{id28}{28}]. Deshalb sollten sie negative Informationen über sich eher ignorieren und sich stattdessen auf positive Informationen konzentrieren [\hyperlink{id29}{29}]. Die Bevorzugung positiver Informationen schlägt sich darin nieder, dass sie positives Feedback eher annehmen als negatives Feedback [\hyperlink{id30}{30}].
        }\label{item:a3}
    \end{enumerate}

    % F4
    \item{
    Warum führt ein Self"=Improvement"=Motiv zu einem Negativbias?
    }\label{item:f4}
    \begin{enumerate}[label=A\arabic*,start=4]
        \item{
        Sind Personen zu Self"=Improvement motiviert, wollen sie sich also auf der in Frage stehenden Dimension ihres Selbstkonzepts verbessern, dann ist negatives Feedback informativer für Verbesserungen als positives Feedback [\hyperlink{id32}{32}]. Wegen des höheren Informationsgehalts negativen Feedbacks sollten Personen negatives Feedback eher annehmen als positives Feedback [\hyperlink{id33}{33}]. Was die Autoren darunter verstehen, dass Feedback mehr oder weniger informativ für Verbesserungen sein kann, wird im nächsten Abschnitt erörtert.
        }\label{item:a4}
    \end{enumerate}

    % F5
    \item{
    Unter welchen Bedingungen sollte eine Motivation zu Self"=Enhancement entstehen?
    }\label{item:f5}
    \begin{enumerate}[label=A\arabic*,start=5]
        \item{
        Ein Self"=Enhancement"=Motiv sollte dann aktiv werden, wenn Personen keine Möglichkeit wahrnehmen, sich hinsichtlich der in Frage stehenden Dimension ihres Selbstkonzepts zu verbessern [\hyperlink{id38}{38}]. Diese Wahrnehmung wird einerseits dadurch bedingt, wie veränderbar die Selbstkonzeptdimension im Allgemeinen erscheint [\hyperlink{id40}{40}]. So erscheint z. B. die eigene Intelligenz als Veranlagung, auf die man keinen Einfluss hat. Andererseits hängt die Wahrnehmung der Verbesserungsmöglichkeiten vom Untersuchungskontexts ab [\hyperlink{id42}{42}]. Wird Teilnehmern einer Studie bspw. eine Gelegenheit zur Übung in einem in Frage stehenden Fähigkeitsbereich eingeräumt, sollten sie höhere Hoffnungen hegen, sich in diesem Aspekt ihres Selbstkonzepts verbessern zu können.
        }\label{item:a5}
    \end{enumerate}

    % F6
    \item{
    Unter welchen Bedingungen sollte eine Motivation zu Self"=Improvement entstehen?
    }\label{item:f6}
    \begin{enumerate}[label=A\arabic*,start=6]
        \item{
        Ein Self"=Improvement"=Motiv sollte dann aktiv werden, wenn Personen eine Möglichkeit wahrnehmen, sich zu verbessern [\hyperlink{id35}{35}]. Diese Wahrnehmung steigert die Hoffnung, Ist"=Soll"=Diskrepanzen überwinden zu können.
        }\label{item:a6}
    \end{enumerate}

    % F7
    \item{
    Warum sollten Personen zu Self"=Enhancement motiviert sein, wenn sie keine Verbesserungsmöglichkeiten wahrnehmen?
    }\label{item:f7}
    \begin{enumerate}[label=A\arabic*,start=7]
        \item{
        Sehen Personen keine Möglichkeit, sich potenziell zu verbessern, dann ist negatives Feedback keine Informationsquelle für etwaige Verbesserungen, sondern vielmehr eine Bedrohung für ein positives Selbstkonzept [\hyperlink{id36}{36}, \hyperlink{id37}{37}]. Würde man negatives Feedback annehmen, dann hätte man eine negative Selbstwahrnehmung übernommen, die man im Nachhinein nicht mehr verbessern könnte \autocite{dunning_trait_1995}.
        }\label{item:a7}
    \end{enumerate}

    % F8
    \item{
    Warum sollten Personen zu Self"=Improvement motiviert sein, wenn sie Verbesserungsmöglichkeiten wahrnehmen?
    }\label{item:f8}
    \begin{enumerate}[label=A\arabic*,start=8]
        \item{
        Sehen Personen eine Möglichkeit, sich zu verbessern, erscheinen Ist-Soll-Diskrepanzen überwindbar. Aus diesem Grund sollte ein Self"=Improvement-Motiv aktiv werden [\hyperlink{id35}{35}]. Mit anderen Worten: Sobald Verbesserungsversuche potenziell zum Erfolg führen, fassen Personen auch den Entschluss, sich verbessern zu wollen. Die Autoren nennen keinen weiteren Grund dafür, warum Personen diesen Entschluss zu Self"=Improvement automatisch fassen.
        }\label{item:a8}
    \end{enumerate}
\end{enumerate}

Betrachtet man die Theorie als Prozess mit aufeinanderfolgenden Phasen, bilden sich Personen zuerst eine Wahrnehmung der Möglichkeiten, sich zu verbessern. Nehmen sie solche Möglichkeiten wahr, wird ein Self"=Improvement"=Motiv aktiv (\labelcref{item:a6}), was zu einem Negativbias in den Änderungen des Selbstkonzepts führt (\labelcref{item:a2}). Sehen sie hingegen keine Möglichkeit, sich zu verbessern, wird ein Self"=Enhancement"=Motiv aktiv (\labelcref{item:a5}), was einen Positivbias zur Folge hat (\labelcref{item:a1}). Die wahrgenommenen Verbesserungsmöglichkeiten moderieren also – vermittelt über Evaluationsmotive -- die Richtung der asymmetrischen Selbstkonzeptänderungen.

In \cref{fig:abb-4} werden die theoretischen Annahmen formal dargestellt, und zwar als Beziehungen zwischen den Konzepten, die sie enthalten. Das VAST"=Display stellt die Theorie als Prozess dar, der von links nach rechts verläuft. Zwischen der Bildung der Wahrnehmung von Verbesserungsmöglichkeiten und der Aktivierung eines Evaluationsmotiv vermitteln die Konzepte, die in den Antworten \labelcref{item:a7} und \labelcref{item:a8} eine Rolle spielen. Zwischen den aktivierten Motiven und dem Negativ- und Positivbias vermitteln die Konzepte, die in den Antworten \labelcref{item:a3} und \labelcref{item:a4} eine Rolle spielen. Wie schlüssig diese vermittelnden Konzepte die Übergange zwischen den Phasen erklären, wird in den nächsten beiden Abschnitten erörtert.
\begin{figure}[p]
    \centering
    \includegraphics[angle=90, height=0.85\textheight]{report/images/Abbildung4_Theoretische_Annahmen.png}
    \caption[Theoretische Annahmen.]{Theoretische Annahmen. Legende Rel.: n = Naming, i = Conceptual Implication, c = Causation, p = Prediction.}
    \label{fig:abb-4}
\end{figure}

\subsubsection{Der Positivbias und der Negativbias als logische Folgen der Evaluationsmotive}

Betrachtet man die Antworten der Autoren auf die Fragen \labelcref{item:f3} und \labelcref{item:f7}, fällt auf, dass in beiden Antworten die Wahrnehmung, dass negatives Feedback eine Bedrohung für ein positives Selbstkonzept darstellt, eine Rolle spielt. Sie dient zum einen als Erklärung dafür, dass Personen zu Self"=Enhancement motiviert sind, wenn sie keine Verbesserungsmöglichkeiten wahrnehmen (\labelcref{item:a7}). Zum anderen soll sie erklären, warum Personen positives Feedback eher annehmen als negatives Feedback, wenn sie (bereits) zu Self"=Enhancement motiviert sind (\labelcref{item:a3}). Diese Doppelung ist unnötig, weil es im Gegensatz zu Self"=Improvement eigentlich keiner weiteren Begründung dafür bedarf, dass ein Positivbias in den Selbstkonzeptänderungen die Folge eines Self"=Enhancement"=Motivs ist.

Im Falle von Self"=Improvement wird nicht unmittelbar ersichtlich, inwiefern ein Negativbias in den Änderungen des Selbstkonzepts dem Vorsatz dient, sich verbessern zu wollen. Zur Erklärung bedarf es vermittelnder, abstrakter Variablen wie des Informationsgehalts von positivem und negativem Feedback (s.u.). Im Fall von Self"=Enhancement ist es bereits ohne solche Variablen offensichtlich, dass es dem Vorsatz, in einem möglichst positiven Licht dazustehen, dient, sich auf positives Feedback zu konzentrieren. Ein Positivbias ist also die direkte Folge eines Self"=Enhancement"=Motivs, wie in \cref{fig:abb-4} zu sehen ist.

\Textcite{brotzeller_exploring_2025} erklären den Negativbias damit, dass negatives Feedback informativer für Verbesserungsvorhaben ist als positives Feedback (\labelcref{item:a4}), und zwar, weil es auf Gelegenheiten hinweist, sich verbessern zu können. Negatives Feedback legt einen eigenen Schwachpunkt offen, aus dem für einen selbst in der Zukunft negative Konsequenzen erwachsen könnten. Aus dieser Art von Feedback zu lernen, ist also besonders wichtig, weil man auf diese Weise solchen Konsequenzen vorbeugen kann. Personen sollten deshalb gewillt sein, negatives Feedback temporär anzunehmen und darauf aufbauend Verbesserungsversuche zu unternehmen. Aus den Ausführungen der Autoren geht allerdings nicht hervor, was Personen überhaupt dazu bewegen sollte, positives Feedback anzunehmen. Schließlich legt es keine Schwachpunkte offen, sondern weist vielmehr darauf hin, dass man sich sogar unterschätzt hat. In diesem Sinne wäre positives Feedback nicht nur weniger informativ als negatives Feedback, sondern gänzlich uninformativ.

Um diese Unklarheit zu beseitigen, soll versucht werden, die Ausführungen der Autoren umzuformulieren und zu ergänzen. Personen, die zu Self"=Improvement motiviert sind, sollten ein Interesse daran haben, möglichst realistische Selbsteinschätzungen zu entwickeln, damit sie darauf aufbauend möglichst effektive Verbesserungsschritte planen können \autocite{kurman_selfenhancement_2006}. Sie sollten dazu motiviert sein, sowohl positives als auch negatives, diskrepantes Feedback anzunehmen, weil beide Arten von Feedback auf Fehleinschätzungen hinweisen (und in diesem Sinne informativ sind), die auf diese Weise korrigiert werden können. Aus negativem Feedback können Personen lernen, dass sie sich fälschlicherweise überschätzt haben, und aus positivem Feedback, dass sie sich fälschlicherweise unterschätzt haben. Erstere Fehleinschätzung kann negative Konsequenzen nach sich ziehen, weil man durch sie eigene Schwachpunkte übersieht. 
% Um negativen Konsequenzen vorzubeugen, ist es besonders wichtig, realistische Einschätzungen der eigenen Unzulänglichkeiten zu erhalten. 
Realistische Einschätzungen der eigenen Unzulänglichkeiten sind daher wichtig, um negativen Konsequenzen vorzubeugen.
Deshalb sollte man vor allem negativem Feedback Gehör schenken.

\subsubsection{Die Evaluationsmotive als logische Folgen von Verbesserungsmöglichkeiten}

In der Theorie sollte ein Self-Enhancement-Motiv als Reaktion auf die Gefahr für ein positives Selbstkonzept aktiv werden, die von negativem Feedback ausgeht, wenn es keine Möglichkeit gibt, sich zu verbessern (\labelcref{item:a7}). Geht von negativem Feedback eine Bedrohung aus, ist ein effektives Mittel zum Aufrechterhalten eines positiven Selbstbilds, dieses Feedback möglichst zu ignorieren und sich auf positives Feedback zu konzentrieren. Self"=Enhancement ist unter diesen Umständen also ein effektives Evaluationsmotiv. Darüber hinaus ist es hier sogar das effektivste Evaluationsmotiv, weil Self"=Improvement ineffektiv ist, wenn es keine Verbesserungsmöglichkeiten gibt. Etwaige Verbesserungsversuche wären von Vornherein zum Scheitern verurteilt. Sollten Personen dennoch versuchen, sich zu verbessern und als Vorbereitung darauf negatives Feedback annehmen, würden sie ihrer Selbstwahrnehmung womöglich einen dauerhaften Schaden zufügen. Vergleichen Personen die Effektivität beider Motive, sollte ihre Wahl ergo auf Self"=Enhancement fallen.

Nehmen Personen Verbesserungsmöglichkeiten wahr, fallen die Gründe weg, wegen derer Self"=Improvement ineffektiv war. Gleichzeitig bleibt Self"=Enhancement ein effektives Evaluationsmotiv, weil es auch unter diesen Umständen möglich ist, sich auf positives Feedback zu konzentrieren und dadurch eine positive Selbstwahrnehmung aufrechtzuerhalten. Es bedarf also eines positiven Beweggrundes, der Personen zu Self"=Improvement statt zu Self"=Enhancement motiviert. Einen solchen Beweggrund nennen \textcite{brotzeller_exploring_2025} nicht direkt (\labelcref{item:a8}), ihre Erklärung weist hier also eine Lücke auf. Folgt man allerdings einem Literaturverweis, wird man bei \textcite{muller-pinzler_negativity-bias_2019} fündig, die eine sehr ähnliche theoretische Erklärung wie die der Autoren skizzieren. In ihrem Modell haben Personen ein natürliches Bedürfnis nach Selbstverbesserung, weshalb sie immer Self"=Improvement bevorzugen, sobald Verbesserung möglich ist.

Neben dieser Erklärung lässt sich in Einklang mit dem weiter oben geforderten allgemeinen Gesetz argumentieren, dass Personen dem Self"=Improvement"=Motiv den Vorzug geben, weil sie es für das effektivere von beiden Motiven halten, wenn sie Verbesserungsmöglichkeiten wahrnehmen. Personen sollten unter einem Self"=Improvement"=Motiv negatives Feedback vermehrt annehmen (\labelcref{item:a2}). Damit setzen sie sich jedoch dem Risiko aus, dauerhaft negative Selbstwahrnehmungen zu übernehmen, weil Verbesserungsbemühungen auch scheitern können. Mit Self"=Enhancement kann dieses Risiko gemindert werden, weil negatives Feedback insgesamt weniger Beachtung findet. Auf der anderen Seite setzen sich Personen ebenfalls einem Risiko aus, wenn sie negatives Feedback ignorieren: Sie laufen Gefahr, sich weiterhin zu überschätzen, woraus ihnen negative Konsequenzen in der Zukunft erwachsen könnten. Geht man davon aus, dass es langfristig das größere Risiko darstellt, an eigenen Schwachpunkten nicht zu arbeiten, als temporär negative Selbstwahrnehmungen zu übernehmen, wäre es logisch, sich für Self"=Improvement zu entscheiden.

\subsection{Explikation eines formalen Modells}

In einer imaginierten Datensimulation wird eine hypothetische Person X modelliert, die an einer simulierten Studie nach dem oben geschilderten Untersuchungsparadigma teilnimmt. Die Person erhält diskrepantes Feedback einer bestimmten Richtung und Größe. Anhand eines formalen Modells der Theorie muss sich nun errechnen lassen, wie sehr sie ihr Selbstkonzept in Richtung dieses Feedbacks anpassen wird. Ein entscheidender Eingangsparameter ist dabei, ob die Person Verbesserungsmöglichkeiten wahrnimmt oder nicht. Das wird zu Beginn der Datensimulation einmal entschieden, was der Einteilung in die experimentellen Gruppen \enquote{low vs. high opportunity for improvement} in Studie 4 bei \textcite{brotzeller_exploring_2025} gleichkommt.

Im Verlauf der weiteren Simulation durchläuft Person X nun die nächsten beiden Phasen des Prozesses, den die Theorie modelliert. Abhängig von der Wahrnehmung der Verbesserungsmöglichkeiten muss zunächst bestimmt werden, welches Evaluationsmotiv bei ihr aktiviert wird. Anschließend muss je nach aktiviertem Motiv und Richtung der Diskrepanz bestimmt werden, wie viel von dem Feedback Person X annehmen wird. Für diese beiden Phasen muss das formale Modell Funktionen definieren.

Verfolgt man den in \cref{fig:abb-4} abgebildeten Prozess, wird ersichtlich, dass sich die anfängliche Zuteilung von Person X zu einer der beiden Gruppen \enquote{low vs. high opportunity for improvement} bis zum Ende des Prozesses fortsetzt, weil es keine Überschneidungen zwischen dem Self"=Improvement- und dem Self"=Enhancement"=Pfad gibt. Beide Motive können nur in einer Gruppe aktiv werden, ebenso wie der Negativbias und der Positivbias nur als Folge eines der Motive auftreten können. Sobald man also weiß, welcher Gruppe Person X angehört, weiß man automatisch, ob sie negatives oder positives Feedback vermehrt in ihr Selbstkonzept integrieren wird. Das formale Modell reduziert sich dementsprechend auf eine Funktion, die abbildet, wie genau der jeweilige Bias bei Personen der beiden Gruppen zustande kommt.

Wie genau die konkreten Tendenzen, positives oder negatives Feedback anzunehmen, aus einem Motiv hervorgehen, wird in der Theorie nicht beschrieben und kann deshalb nicht modelliert werden. Die Datensimulation kann nicht fortgeführt werden, weil nicht berechnet werden kann, wie viel des Feedbacks Person X nun tatsächlich annehmen wird. Auf dem Self"=Improvement"=Pfad ließen sich diese Tendenzen theoretisch als Funktion des Informationsgehalts der jeweiligen Art des Feedbacks errechnen. Diese Zusammenhänge werden in der Theorie der Autoren jedoch nicht hergestellt.

Das formale Modell kann also nicht erklären, wie der Negativ- und der Positivbias bei bestimmten Wahrnehmungen von Verbesserungsmöglichkeiten zustande kommen. Der Definition produktiver Erklärungen zufolge besitzt die Theorie von \textcite{brotzeller_exploring_2025} eine geringe Erklärungskraft, weil das formale Modell der Theorie das Phänomen in einer Datensimulation nicht produzieren kann. Das formale Modell enthält keine mathematischen Funktionen, sondern besteht lediglich aus einer qualitativen Funktion, die der Gruppe, die Verbesserungsmöglichkeiten wahrnimmt, einen Negativbias zuordnet, der Gruppe, die keine solchen Möglichkeiten wahrnimmt, hingegen einen Positivbias. Damit ist das Modell lediglich eine Wiederholung der formalen Definition des Phänomens, das im unteren Kasten in \cref{fig:placeholder-3} gezeigt wurde.

% \subsection{Ein subjektiver Schwellenwert für die Wahrnehmung von Verbesserungsmöglichkeiten}
% \label{sec:search-thresh}

% \subsection{Die Evaluationsmotive als logische Folgen niedriger und hoher Verbesserungsmöglichkeiten}
% \label{sec:search-eval}

% \subsection{Der Positivbias und der Negativbias als logische Folgen der Evaluationsmotive}
% \label{sec:search-bias}
