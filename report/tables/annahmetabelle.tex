\begin{xltabular}{\linewidth}{l >{\RaggedRight\hsize=1.4\hsize}X >{\RaggedRight\hsize=0.9\hsize}X c >{\RaggedRight\hsize=0.7\hsize}X r}

    % --- Tabellenkopf (erste Seite) ---
    \caption{Theoretische Annahmen und Konzepte nach \autocite{brotzeller_exploring_2025}.\label{tab:annahmetabelle}} \\
    \toprule
    \textbf{Seite} & \textbf{Zitat} & \textbf{Annahme} & \textbf{Rel.} & \textbf{Konzepte} & \textbf{ID} \\
    \midrule
    \endfirsthead

    % --- Tabellenkopf (Folgeseiten) ---
    \caption[]{Theoretische Annahmen (Fortsetzung)} \\
    \toprule
    \textbf{Seite} & \textbf{Zitat} & \textbf{Annahme} & \textbf{Rel.} & \textbf{Konzepte} & \textbf{ID} \\
    \midrule
    \endhead

    % --- Tabellenfuß (Folgeseiten) ---
    \midrule
    \multicolumn{6}{r}{\textit{Fortsetzung auf der nächsten Seite...}} \\
    \endfoot

    % --- Tabellenfuß (letzte Seite) ---
    \bottomrule
    \multicolumn{6}{p{\linewidth}}{%
        \footnotesize \textit{Hinweis.} Alle Zitate stammen aus \textcite{brotzeller_exploring_2025}. Aus jedem Zitat wurden die einzelnen Annahmen extrahiert, die sie enthält. Die Annahmen 1--18 betreffen Definitionen von Phänomenen und Variablen, die Annahmen 19--45 hingegen die Annahmen der von den Autoren dargestellten Theorie. Jeder Annahme werden die Konzepte zugeordnet, die sie in Beziehung setzt. Ein Higher"=Order"=Concept (HOC) umschließt die darauffolgenden Konzepte. Legende Rel.: n = Naming, i = Conceptual Implication, c = Causation, p = Prediction
    } \\
    \endlastfoot

    % --- INHALT ---
    
    1731 & \multirow{3}{=}{In both settings, self-relevant feedback can shape people’s self-concept, defined as a person’s perception of themselves (Bem, 1972; Shavelson et al., 1976). However, the extent to which discrepant external feedback leads to changes in such self-perceptions (i.e., self-concept change) varies considerably: In some cases, receiving external feedback leads to self-concept change in accordance with the feedback, while, in other cases, even highly discrepant feedback does not lead to self-concept change. [\dots] Although a considerable amount of research has examined self-concept change after self-relevant feedback, many unanswered questions remain.}
         & Finding: Self-concept change after self-relevant feedback 
         & n & SELF-CONCEPT CHANGE AFTER SELF-RELEVANT FEEDBACK & \hypertarget{id1}{1} \\
    % \addlinespace
         & % Zitat wird nicht wiederholt
         & Finding: Discrepant external feedback leads to changes in self-perceptions (i.e., self-concept change) 
         & c & HOC: SELF-CONCEPT CHANGE AFTER SELF-RELEVANT FEEDBACK, DISCREPANT FEEDBACK, SELF-CONCEPT CHANGE & \hypertarget{id2}{2} \\
    % \addlinespace
         & 
         & Finding: Self-concept change in accordance with the feedback 
         & n & SELF-CONCEPT CHANGE AFTER SELF-RELEVANT FEEDBACK & \hypertarget{id3}{3} \\
    \midrule

    1731 & \multirow{2}{=}{In the present research, we focus on the latter of the three features [i.e., features of the feedback itself] and examine the size of the discrepancy between one’s self-view and the feedback one receives as well as the direction of the discrepancy -- that is, whether the feedback is positive and suggests an upward adjustment of one’s self-concept (e.g., “I obviously cook better than I thought I would”) or whether it is negative and suggests a downward adjustment (e.g., “I obviously cook worse than I thought I would”).}
         & Size of discrepancy between one’s self-view and the feedback one receives
         & n & SIZE OF DISCREPANCY & \hypertarget{id4}{4} \\
    \addlinespace
         & 
         & Direction of the discrepancy between one’s self-view and the feedback one receives
         & n & DIRECTION OF DISCREPANCY & \hypertarget{id5}{5} \\\\\\\\\\\\
    \midrule

    1732 & \multirow{2}{=}{Studies investigating the effect of the size of the discrepancy between self-concept and feedback consistently demonstrate that larger discrepancies lead to more self-concept change (i.e., larger differences between previous and current self-perceptions) than smaller discrepancies, except for extreme and likely implausible discrepancies (Bergin, 1962; Binderman et al., 1972; Kube et al., 2022).}
         & Finding: Larger discrepancies between self-concept and feedback lead to more self-concept change 
         & p, c & SIZE OF DISCREPANCY, SELF-CONCEPT CHANGE & \hypertarget{id6}{6} \\
    \addlinespace
         & 
         & Finding: Extreme and likely implausible discrepancies do not lead to self-concept change 
         & p 0 & SIZE OF DISCREPANCY, SELF-CONCEPT CHANGE & \hypertarget{id7}{7} \\\\\\
    \midrule

    1732 & \multirow{3}{=}{Regarding the direction of the discrepancy, however, the empirical findings are less conclusive. Several studies show that positive and negative feedback lead to different amounts of self-concept change. [\dots] 
    % Interestingly, it is unclear which of the two types of feedback leads to larger self-concept change: 
    The majority of studies find larger self-concept change after positive than after negative feedback, indicating a positivity bias in the processing of self-relevant information (Eil \& Rao, 2011; Elder et al., 2022; Korn et al., 2012; Möbius et al., 2022). [\dots]
    %The term positivity bias hereby is not meant to imply that such processing of self-relevant information is irrational; we merely use it to describe cases in which positive feedback produces more self-concept change than negative feedback. 
    Notably, two recent studies demonstrate larger self-concept change after negative than after positive feedback (Ertac, 2011; Müller-Pinzler et al., 2019) -- a pattern that rather suggests a negativity bias.}
         & Positive and negative feedback lead to different amounts of self-concept change 
         & p & HOC: ASYMMETRIC SELF-CONCEPT CHANGE, DIRECTION OF DISCREPANCY, SELF-CONCEPT CHANGE & \hypertarget{id8}{8} \\
    \addlinespace
         & 
         & Positivity bias in self"=concept change
         & n & POSITIVITY BIAS & \hypertarget{id9}{9} \\
    \addlinespace
         & 
         & Positive feedback produces more self-concept change than negative feedback 
         & p & HOC: POSITIVITY BIAS, DIRECTION OF DISCREPANCY, SELF-CONCEPT CHANGE & \hypertarget{id10}{10} \\\\\\
    \addlinespace
         & 
         & Larger self"=concept change after negative than after positive feedback 
         & p & HOC: NEGATIVITY BIAS, DIRECTION OF DISCREPANCY, SELF-CONCEPT CHANGE & \hypertarget{id11}{11} \\
    \addlinespace
         & 
         & Negativity bias in self"=concept change
         & n & NEGATIVITY BIAS & \hypertarget{id12}{12} \\
    \midrule
    1732 & 1732	While we are mainly interested in the main effects of size and direction of discrepancy separately, we also explore whether they interact in producing self-concept change. Prior studies have largely neglected the possible interaction effects of these variables. & The size and direction of discrepancy might interact in producing self-concept change & p ? & SIZE OF DISCREPANCY, DIRECTION OF DISCREPANCY,
SELF"=CONCEPT CHANGE & \hypertarget{id13}{13} \\\\
    \midrule


    1732 & \multirow{4}{=}{Our own definition of self-concept change builds upon the definition by Shavelson et al. (1976) and attempts to be even more precise: We argue that self-concept change has occurred whenever a person’s perception of themselves on a specific self-relevant dimension at a given time point differs from a previous self-perception on the same dimension.}
         & Self-concept change 
         & n & SELF-CONCEPT CHANGE & \hypertarget{id14}{14} \\
    \addlinespace
         & 
         & Self-perception on a specific self-relevant dimension at time point t1 
         & n & SELF-PERCEPTION & \hypertarget{id15}{15} \\
    \addlinespace
         & 
         & Self-perception on a specific self-relevant dimension at time point t2 
         & n & SELF-PERCEPTION & \hypertarget{id16}{16} \\
    \addlinespace
         & 
         & Self-concept change has occurred when $\mathrm{self-perception}_{\mathrm{t1}}$ differs from $\mathrm{self-perception}_{\mathrm{t2}}$
         & i & SELF-PERCEPTION, DIFFERENCE IN SELF-PERCEPTIONS, SELF-CONCEPT CHANGE & \hypertarget{id17}{17} \\
    \midrule

    1732 & Feedback means any kind of external information that a person receives on a self-relevant dimension (e.g., on a trait or an ability). Importantly, this feedback must be perceived as diagnostically relevant for this specific dimension: [\dots]
         & External, self-relevant feedback 
         & n & DISCREPANT FEEDBACK & \hypertarget{id18}{18} \\
    \midrule

    1733 & \multirow{2}{=}{There are different theoretical approaches to explaining the positivity and negativity bias in self-concept change and the contradictory findings that have resulted from previous research. One such approach focuses on two processes that shape how people perceive and integrate feedback into the self-concept: self-enhancement and self-improvement.}
         & Perception of feedback 
         & n & PERCEPTION OF FEEDBACK & \hypertarget{id19}{19} \\
    \addlinespace
         & 
         & Self-enhancement processes shape how people perceive and integrate feedback into the self-concept 
         & c & SELF-ENHANCEMENT PROCESSES, PERCEPTION OF FEEDBACK, SELF-CONCEPT CHANGE & \hypertarget{id20}{20} \\\\
    \addlinespace
         & 
         & Self-improvement processes shape how people perceive and integrate feedback into the self-concept 
         & c & SELF-IMPROVEMENT PROCESSES, PERCEPTION OF FEEDBACK, SELF-CONCEPT CHANGE & \hypertarget{id21}{21} \\
    \midrule

    1733 & \multirow{2}{=}{While self-enhancement describes biases in processing and interpreting information in a self-serving fashion (Heine \& Hamamura, 2007), self-improvement describes biases aimed at reducing discrepancies between an “is-state” and a desirable “ought-state” (Kurmann, 2006). }
         & Self-enhancement processes are biases in processing and interpreting information in a self-serving fashion 
         & n & SELF-ENHANCEMENT PROCESSES & \hypertarget{id22}{22} \\
    \addlinespace
         & 
         & Self-Improvement processes are biases aimed at approaching a desirable ought-state 
         & n & SELF-IMPROVEMENT PROCESSES & \hypertarget{id23}{23} \\\\
    \midrule

    1733 & \multirow{3}{=}{Both self-enhancement and self-improvement assume that people are motivated to maintain a positive view of themselves even (or particularly) in the face of disconfirming feedback (Taylor \& Brown, 1988).}
         & Motivation to maintain a positive view of oneself 
         & n & POSITIVE SELF MOTIVE & \hypertarget{id24}{24} \\
    \addlinespace
         & 
         & Self-enhancement assumes a motivation to maintain a positive view of oneself 
         & i & SELF-ENHANCEMENT MOTIVE, POSITIVE SELF MOTIVE & \hypertarget{id25}{25} \\
    \addlinespace
         & 
         & Self-improvement assumes a motivation to maintain a positive view of oneself 
         & i & SELF-IMPROVEMENT MOTIVE, POSITIVE SELF MOTIVE & \hypertarget{id26}{26} \\\\\\\\
    \midrule

    1733 & \multirow{3}{=}{When a person receiving feedback is motivated to self-enhance, they should focus on positive and dismiss negative information as the latter is perceived as threatening one’s positive self-view. Therefore, a self-enhancement motive should lead to positively biased self-concept change.}
         & Motivation to self-enhance 
         & n & SELF-ENHANCEMENT MOTIVE & \hypertarget{id27}{27}\\
    \addlinespace
         & 
         & When a person is motivated to self-enhance, negative discrepant feedback is perceived as threatening one’s positive self-view 
         & i & SELF-ENHANCEMENT MOTIVE, HOC: PERCEPTION OF FEEDBACK, THREAD BY NEGATIVE FEEDBACK & \hypertarget{id28}{28} \\
    \addlinespace
         & 
         & People should focus on positive and dismiss negative information as the latter is perceived as threatening one’s self-view 
         & c & THREAD BY NEGATIVE FEEDBACK, FOCUS ON POSITIVE FEEDBACK & \hypertarget{id29}{29} \\
    \addlinespace
         & 
         & Because people focus on positive information as negative information is perceived as threatening under a self-enhancement motive, a self-enhancement motive should trigger a positivity bias in self-concept change
         & c & SELF-ENHANCEMENT MOTIVE, THREAD BY NEGATIVE FEEDBACK, FOCUS ON POSITIVE FEEDBACK, POSITIVITY BIAS & \hypertarget{id30}{30} \\
    \midrule

    1733 & When a person receiving feedback is motivated to self-improve, however, negative feedback is more informative than positive feedback because the former highlights opportunities for improvement. In other words, such a person should be negatively biased in changing their self-concept.
         & Motivation to self-improve 
         & n & SELF-IMPROVEMENT MOTIVE & \hypertarget{id31}{31} \\
    \addlinespace
         & 
         & If a person is motivated to self-improve, negative discrepant feedback is more informative than positive 
         & i & SELF-IMPROVEMENT MOTIVE, HOC: PERCEPTION OF FEEDBACK, INFORMATIONAL VALUE OF [NEGATIVE|POSITIVE] FEEDBACK & \hypertarget{id32}{32} \\
    \addlinespace
         & 
         & The perception that negative feedback is more informative (for improvement) under a self-improvement motive should lead to a negativity bias in self-concept change
         & c, i & SELF-IMPROVEMENT MOTIVE, INFORMATIVITY OF FEEDBACK, NEGATIVITY BIAS & \hypertarget{id33}{33} \\
    \midrule

    1733 & \multirow{3}{=}{
    While self-improvement is triggered in particular when a person perceives that they can overcome is-ought discrepancies (e.g., by practicing or rehearsing), 
    self-enhancement should be triggered when a person perceives it as impossible to improve on the self-concept aspect in question (Müller-Pinzler et al., 2019). 
    When the aspect of the self-concept is perceived as fixed and unimprovable, negative feedback does not have an informational value toward improving oneself [\dots].
    but is, instead, particularly threatening to one’s positive self-view (Dunning, 1995; Dweck et al., 1995; Levy \& Dweck, 1998). 
    In such cases, the only possibility of maintaining one’s positive 
    % self-view is to self-enhance. Perceiving little opportunity for improvement should therefore trigger self-enhancement and produce positively biased self-concept change.
    }
         & Perceived possibility to improve on the self-concept aspect in question 
         & n & OPPORTUNITY FOR IMPROVEMENT & \hypertarget{id34}{34} \\
    \addlinespace
         & 
         & Perceiving a possibility to improve, i.e. a possibility to overcome is-ought discrepancies, triggers a self-improvement motivation 
         & c & OPPORTUNITY FOR IMPROVEMENT, SELF-IMPROVEMENT MOTIVE & \hypertarget{id35}{35} \\
    \addlinespace
         & 
         & If people perceive it as impossible to improve, negative feedback has no informational value 
         & i -1 & OPPORTUNITY FOR IMPROVEMENT, INFORMATIONAL VALUE OF NEGATIVE FEEDBACK & \hypertarget{id36}{36} \\\\
         
    \addlinespace
         & \multirow{2}{=}{
         self-view is to self-enhance. Perceiving little opportunity for improvement should therefore trigger self-enhancement and produce positively biased self-concept change.
         }
         & If people perceive it as impossible to improve, negative feedback is particularly threatening to one’s self-view 
         & i & OPPORTUNITY FOR IMPROVEMENT, THREAD BY NEGATIVE FEEDBACK & \hypertarget{id37}{37} \\
    \addlinespace
         & 
         & If negative feedback has no informational value but is perceived as threatening instead, self-enhancement is the only possibility of maintaining one’s positive self-view. Therefore, perceiving little opportunity for improvement triggers self-enhancement.
         & c, i & OPPORTUNITY FOR IMPROVEMENT, INFORMATIONAL VALUE OF NEGATIVE FEEDBACK, THREAD BY NEGATIVE FEEDBACK, SELF-ENHANCEMENT MOTIVE & \hypertarget{id38}{38} \\
    \midrule

    1733 & \multirow{4}{=}{Supporting this theorizing, a positivity bias—reflecting a self-enhancement process—has been empirically demonstrated on those self-concept aspects that are most likely to be perceived as fixed and unchangeable by most people (e.g., intelligence or beauty, see Eil \& Rao, 2011; Möbius et al., 2022) or if the study was designed such that participants likely saw little opportunity for improvement (e.g., one-shot feedback from third parties; see Elder et al., 2022; Korn et al., 2012). These findings are also consistent with other studies on belief updating after feedback (Lefebvre et al., 2017).}
         & Modifiability of the dimension of the self-concept in question 
         & n & MALLEABILITY & \hypertarget{id39}{39} \\
    \addlinespace
         & 
         & Perceiving that the self-concept aspect in question is likely to be modifiable promotes the perception that it is possible to improve on this aspect 
         & i & MALLEABILITY, OPPORTUNITY FOR IMPROVEMENT & \hypertarget{id40}{40} \\
    \addlinespace
         & 
         & Design of an empirical study 
         & n & STUDY DESIGN & \hypertarget{id41}{41} \\
    \addlinespace
         & 
         & Certain features of the study design promote or diminish the perceived opportunity for improvement 
         & i & STUDY DESIGN, OPPORTUNITY FOR IMPROVEMENT & \hypertarget{id42}{42} \\\\
    \midrule

    1733 & By contrast, when the self-concept aspect in question is perceived as improvable (“malleable”), negative feedback is more informative for self-improvement purposes than positive feedback (Strube, 2012). 
         & If the self-concept aspect in question is perceived as improvable, negative feedback is more informative for self-improvement purposes than positive feedback 
         & i & MALLEABILITY, INFORMATIONAL VALUE OF NEGATIVE FEEDBACK, INFORMATIONAL VALUE OF POSITIVE FEEDBACK & \hypertarget{id43}{43} \\\\\\\\\\\\\\\\
    \midrule
    1733 & \multirow{2}{=}{To sum up, the opportunity for improvement in conjunction with motives for self-enhancement and self-improvement may be a plausible explanation for the contradictory findings on self-concept change after negative vs. positive feedback. Yet, this explanation has not been systematically examined so far. Therefore, the present research investigates the role of the opportunity for improvement in asymmetric self-concept change.} & Contradictory findings on self-concept change after negative vs. positive feedback & n & CONTRADICTORY FINDINGS & \hypertarget{id44}{44}\\
    \addlinespace
    &
    & Asymmetric self-concept change & n & ASYMMETRIC SELF-CONCEPT CHANGE & \hypertarget{id45}{45}\\\\\\\\\\\\

\end{xltabular}