\begin{xltabular}{\linewidth}{l >{\RaggedRight\hsize=1.4\hsize}X >{\RaggedRight\hsize=0.9\hsize}X c >{\RaggedRight\hsize=0.7\hsize}X r}

    % --- Tabellenkopf (erste Seite) ---
    \caption{Theoretische Annahmen und Konzepte nach \autocite{brotzeller_exploring_2025}.\label{tab:annahmetabelle}} \\
    \toprule
    \textbf{Seite} & \textbf{Zitat} & \textbf{Annahme} & \textbf{Rel.} & \textbf{Konzepte} & \textbf{ID} \\
    \midrule
    \endfirsthead

    % --- Tabellenkopf (Folgeseiten) ---
    \caption[]{Theoretische Annahmen (Fortsetzung)} \\
    \toprule
    \textbf{Seite} & \textbf{Zitat} & \textbf{Annahme} & \textbf{Rel.} & \textbf{Konzepte} & \textbf{ID} \\
    \midrule
    \endhead

    % --- Tabellenfuß (Folgeseiten) ---
    \midrule
    \multicolumn{6}{r}{\textit{Fortsetzung auf der nächsten Seite...}} \\
    \endfoot

    % --- Tabellenfuß (letzte Seite) ---
    \bottomrule
    \multicolumn{6}{p{\linewidth}}{%
        \footnotesize \textit{Hinweis.} Alle Zitate stammen aus \autocite{brotzeller_exploring_2025}. Aus den jeweiligen Zitaten werden die einzelnen Aussagen bzw. Annahmen extrahiert. Die Annahmen 1–18 betreffen Definitionen; Annahmen 18–42 betreffen die Theorie. Legende Rel.: n = nominell, c = kausal, i = inferenziell, p = positiv korreliert.
    } \\
    \endlastfoot

    % --- INHALT ---
    
    1731 & In both settings, self-relevant feedback can shape people’s self-concept [...] even highly discrepant feedback does not lead to self-concept change. [...] many unanswered questions remain. 
         & Finding: Self-concept change after self-relevant feedback 
         & n & SELF-CONCEPT CHANGE AFTER SELF-RELEVANT FEEDBACK & 1 \\
    \addlinespace
         & % Zitat wird nicht wiederholt
         & Finding: Discrepant external feedback leads to changes in self-perceptions (i.e., self-concept change) 
         & c & HOC: SELF-CONCEPT CHANGE AFTER SELF-RELEVANT FEEDBACK, DISCREPANT FEEDBACK, SELF-CONCEPT CHANGE & 2 \\
    \addlinespace
         & 
         & Finding: Self-concept change in accordance with the feedback 
         & n & SELF-CONCEPT CHANGE AFTER SELF-RELEVANT FEEDBACK & 3 \\
    \midrule

    1731 & In the present research, we focus on the latter of the three features [...] whether the feedback is positive [...] or whether it is negative [...]. 
         & Size of discrepancy between one’s self-view and the feedback one receives 
         & n & SIZE OF DISCREPANCY & 4 \\
    \addlinespace
         & 
         & Direction of the discrepancy between one’s self-view and the feedback one receives 
         & n & DIRECTION OF DISCREPANCY & 5 \\
    \midrule

    1732 & Studies investigating the effect of the size of the discrepancy [...] consistently demonstrate that larger discrepancies lead to more self-concept change [...] except for extreme and likely implausible discrepancies. 
         & Finding: Larger discrepancies between self-concept and feedback lead to more self-concept change 
         & p, c & SIZE OF DISCREPANCY, SELF-CONCEPT CHANGE & 6 \\
    \addlinespace
         & 
         & Finding: Extreme and likely implausible discrepancies do not lead to self-concept change 
         & p 0 & SIZE OF DISCREPANCY, SELF-CONCEPT CHANGE & 7 \\
    \midrule

    1732 & Regarding the direction of the discrepancy, however, the empirical findings are less conclusive. [...] The majority of studies find larger self-concept change after positive than after negative feedback [...] Notably, two recent studies demonstrate larger self-concept change after negative than after positive feedback [...] 
         & Positive and negative feedback lead to different amounts of self-concept change 
         & p & HOC: ASYMMETRIC SELF-CONCEPT CHANGE, DIRECTION OF DISCREPANCY, SELF-CONCEPT CHANGE & 8 \\
    \addlinespace
         & 
         & Positivity bias in the processing of self-relevant feedback 
         & n & POSITIVITY BIAS & 9 \\
    \addlinespace
         & 
         & Positive feedback produces more self-concept change than negative feedback 
         & p & HOC: POSITIVITY BIAS, DIRECTION OF DISCREPANCY, SELF-CONCEPT CHANGE & 10 \\
    \addlinespace
         & 
         & Larger self-concept change after negative than after positive feedback 
         & p & HOC: NEGATIVITY BIAS, DIRECTION OF DISCREPANCY, SELF-CONCEPT CHANGE & 11 \\
    \addlinespace
         & 
         & Negativity bias 
         & n & NEGATIVITY BIAS & 12 \\
    \midrule

    1732 & Our own definition of self-concept change builds upon the definition by Shavelson et al. (1976) [...] We argue that self-concept change has occurred whenever a person’s perception of themselves [...] differs from a previous self-perception on the same dimension. 
         & Self-concept change 
         & n & SELF-CONCEPT CHANGE & 13 \\
    \addlinespace
         & 
         & Self-perception on a specific self-relevant dimension at time point t1 
         & n & SELF-PERCEPTION & 14 \\
    \addlinespace
         & 
         & Self-perception on a specific self-relevant dimension at time point t2 
         & n & SELF-PERCEPTION & 15 \\
    \addlinespace
         & 
         & Self-concept change has occurred when self-perception t1 differs from self-perception t2 
         & i & SELF-PERCEPTION, DIFFERENCE IN SELF-PERCEPTIONS, SELF-CONCEPT CHANGE & 16 \\
    \midrule

    1732 & Feedback means any kind of external information that a person receives on a self-relevant dimension [...] Importantly, this feedback must be perceived as diagnostically relevant [...] 
         & External, self-relevant feedback 
         & n & DISCREPANT FEEDBACK & 17 \\
    \midrule

    1733 & There are different theoretical approaches to explaining the positivity and negativity bias [...] One such approach focuses on two processes [...] self-enhancement and self-improvement. 
         & Perception of feedback 
         & n & PERCEPTION OF FEEDBACK & 18 \\
    \addlinespace
         & 
         & Self-enhancement processes shape how people perceive and integrate feedback into the self-concept 
         & c & SELF-ENHANCEMENT PROCESSES, PERCEPTION OF FEEDBACK, SELF-CONCEPT CHANGE & 19 \\
    \addlinespace
         & 
         & Self-improvement processes shape how people perceive and integrate feedback into the self-concept 
         & c & SELF-IMPROVEMENT PROCESSES, PERCEPTION OF FEEDBACK, SELF-CONCEPT CHANGE & 20 \\
    \midrule

    1733 & While self-enhancement describes biases in processing and interpreting information in a self-serving fashion [...] self-improvement describes biases aimed at reducing discrepancies between an “is-state” and a desirable “ought-state”. 
         & Self-enhancement processes are biases in processing and interpreting information in a self-serving fashion 
         & n & SELF-ENHANCEMENT PROCESSES & 21 \\
    \addlinespace
         & 
         & Self-Improvement processes are biases aimed at approaching a desirable ought-state 
         & n & SELF-IMPROVEMENT PROCESSES & 22 \\
    \midrule

    1733 & Both self-enhancement and self-improvement assume that people are motivated to maintain a positive view of themselves [...] 
         & Motivation to maintain a positive view of oneself 
         & n & POSITIVE SELF MOTIVE & 23 \\
    \addlinespace
         & 
         & Self-enhancement assumes a motivation to maintain a positive view of oneself 
         & i & SELF-ENHANCEMENT MOTIVE, POSITIVE SELF MOTIVE & 24 \\
    \addlinespace
         & 
         & Self-improvement assumes a motivation to maintain a positive view of oneself 
         & i & SELF-IMPROVEMENT MOTIVE, POSITIVE SELF MOTIVE & 25 \\
    \midrule

    1733 & When a person receiving feedback is motivated to self-enhance, they should focus on positive and dismiss negative information [...] Therefore, a self-enhancement motive should lead to positively biased self-concept change. 
         & Motivation to self-enhance 
         & n & SELF-ENHANCEMENT MOTIVE & 26 \\
    \addlinespace
         & 
         & When a person is motivated to self-enhance, negative discrepant feedback is perceived as threatening one’s positive self-view 
         & i & SELF-ENHANCEMENT MOTIVE, HOC: PERCEPTION OF FEEDBACK, THREAD BY NEGATIVE FEEDBACK & 27 \\
    \addlinespace
         & 
         & People should focus on positive and dismiss negative information as the latter is perceived as threatening one’s self-view 
         & c & THREAD BY NEGATIVE FEEDBACK, FOCUS ON POSITIVE FEEDBACK & 28 \\
    \addlinespace
         & 
         & Because people focus on positive information [...] a self-enhancement motive should trigger a positivity bias in self-concept change 
         & c & SELF-ENHANCEMENT MOTIVE, THREAD BY NEGATIVE FEEDBACK, FOCUS ON POSITIVE FEEDBACK, POSITIVITY BIAS & 29 \\
    \midrule

    1733 & When a person receiving feedback is motivated to self-improve, however, negative feedback is more informative than positive feedback [...] such a person should be negatively biased in changing their self-concept. 
         & Motivation to self-improve 
         & n & SELF-IMPROVEMENT MOTIVE & 30 \\
    \addlinespace
         & 
         & If a person is motivated to self-improve, negative discrepant feedback is more informative than positive 
         & i & SELF-IMPROVEMENT MOTIVE, HOC: PERCEPTION OF FEEDBACK, INFORMATIONAL VALUE OF NEGATIVE FEEDBACK & 31 \\
    \addlinespace
         & 
         & The perception that negative feedback is more informative [...] should lead to a negativity bias in self-concept change 
         & c, i & SELF-IMPROVEMENT MOTIVE, INFORMATIVITY OF FEEDBACK, NEGATIVITY BIAS & 32 \\
    \midrule

    1733 & While self-improvement is triggered in particular when a person perceives that they can overcome is-ought discrepancies [...], self-enhancement should be triggered when a person perceives it as impossible to improve [...] 
         & Perceived possibility to improve on the self-concept aspect in question 
         & n & OPPORTUNITY FOR IMPROVEMENT & 33 \\
    \addlinespace
         & 
         & Perceiving a possibility to improve, i.e. a possibility to overcome is-ought discrepancies, triggers a self-improvement motivation 
         & c & OPPORTUNITY FOR IMPROVEMENT, SELF-IMPROVEMENT MOTIVE & 34 \\
    \addlinespace
         & 
         & If people perceive it as impossible to improve, negative feedback has no informational value 
         & i -1 & OPPORTUNITY FOR IMPROVEMENT, INFORMATIONAL VALUE OF NEGATIVE FEEDBACK & 35 \\
    \addlinespace
         & 
         & If people perceive it as impossible to improve, negative feedback is particularly threatening to one’s self-view 
         & i & OPPORTUNITY FOR IMPROVEMENT, THREAD BY NEGATIVE FEEDBACK & 36 \\
    \addlinespace
         & 
         & If negative feedback has no informational value but is perceived as threatening instead, self-enhancement is the only possibility [...] 
         & c, i & OPPORTUNITY FOR IMPROVEMENT, INFORMATIONAL VALUE OF NEGATIVE FEEDBACK, THREAD BY NEGATIVE FEEDBACK, SELF-ENHANCEMENT MOTIVE & 37 \\
    \midrule

    1733 & Supporting this theorizing, a positivity bias [...] has been empirically demonstrated on those self-concept aspects that are most likely to be perceived as fixed [...] or if the study was designed such that participants likely saw little opportunity for improvement [...] 
         & Modifiability of the dimension of the self-concept in question 
         & n & MALLEABILITY & 38 \\
    \addlinespace
         & 
         & Perceiving that the self-concept aspect in question is likely to be modifiable promotes the perception that it is possible to improve on this aspect 
         & i & MALLEABILITY, OPPORTUNITY FOR IMPROVEMENT & 39 \\
    \addlinespace
         & 
         & Design of an empirical study 
         & n & STUDY DESIGN & 40 \\
    \addlinespace
         & 
         & Certain features of the study design promote or diminish the perceived opportunity for improvement 
         & i & STUDY DESIGN, OPPORTUNITY FOR IMPROVEMENT & 41 \\
    \midrule

    1733 & By contrast, when the self-concept aspect in question is perceived as improvable (“malleable”), negative feedback is more informative for self-improvement purposes than positive feedback (Strube, 2012). 
         & If the self-concept aspect in question is perceived as improvable, negative feedback is more informative for self-improvement purposes than positive feedback 
         & i & MALLEABILITY, INFORMATIONAL VALUE OF NEGATIVE FEEDBACK, INFORMATIONAL VALUE OF POSITIVE FEEDBACK & 42 \\

\end{xltabular}